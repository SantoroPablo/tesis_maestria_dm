\documentclass[spanish]{article}
\usepackage[utf8]{inputenc}
\usepackage[T1]{fontenc}
\usepackage{graphicx}
\usepackage[spanish, english]{babel}

\graphicspath{ {./images/} }

\begin{document}

\begin{titlepage}
    \begin{center}

    \huge
    \textbf{Plan de Tesis de Maestría en Explotación de Datos y Descubrimiento del Conocimiento}
    
    \vspace{0.5cm}
    
    \vfill
    \Large
    Tesis a ser presentada para obtener el título de\\
    Magíster en Explotación de Datos y Descubrimiento del Conocimiento
    
    \vspace{0.5cm}
    
    \includegraphics[width=0.4\textwidth]{images/logo_uba.png}
    
    \vspace{0.5cm}
    \large
    Facultad de Ciencias Exactas y Naturales\\
    Maestría en Explotación de Datos y Descubrimiento del Conocimiento\\
    
    \vspace{0.5cm}
    
    \includegraphics[width=0.4\textwidth]{images/logo_maestria_dm.png}

    \vspace{0.5cm}
    
    Autor: Pablo Santoro
    
    \vspace{0.5cm}
    
    Director de Tesis:\\
    Dr. Emmanuel Iarussi
    
    {\selectlanguage{spanish}\today}
    
    \end{center}
\end{titlepage}
\thispagestyle{plain}
\begin{center}
    \large
    \textbf{Thesis Title}
    \vspace{0.4cm}
    \textbf{Pablo Santoro}
    \vspace{Resumen}
\end{center}
Lorem ipsum dolor sit amet, consectetur adipiscing elit. Cras at ullamcorper ligula. Fusce varius nibh dui, in semper lacus semper finibus. Donec ut orci eget nibh efficitur fringilla nec vel mi. Morbi ornare nunc vitae orci dictum, sed malesuada nisi accumsan. Curabitur ultrices vel urna sed tincidunt. Proin at cursus quam. Aenean pulvinar convallis vehicula. Mauris hendrerit ac ligula quis cursus. Sed at lorem vel sapien porttitor volutpat quis id libero.

Fusce porta est pellentesque, suscipit diam id, ultrices tellus. Nulla feugiat turpis erat, non mattis erat tincidunt non. Praesent vitae urna eu augue accumsan placerat non quis nulla. Mauris euismod dolor quis accumsan luctus. Etiam dolor mauris, consectetur pharetra tortor sed, dapibus scelerisque orci. Integer eu lectus sagittis, imperdiet ligula id, tempor justo. Etiam nec lacus ut erat vulputate hendrerit. Proin eleifend hendrerit vehicula. Etiam molestie ipsum eu metus hendrerit bibendum accumsan nec sapien. Proin mi dolor, lobortis ac ultrices ac, porta non sem. Nam vel nibh eget nunc cursus efficitur. Mauris eu elit lacinia, pulvinar tellus sit amet, accumsan leo. Nam venenatis arcu ipsum, ut rutrum dolor consectetur quis.

Nulla ac ex rutrum, faucibus tortor non, vulputate quam. Pellentesque tempor fringilla odio sed tincidunt. Donec condimentum felis sit amet erat posuere, ut blandit turpis congue. Praesent bibendum eget justo vitae ultrices. Vivamus aliquet tincidunt egestas. Ut at feugiat ante. Phasellus in nunc nec tellus venenatis lacinia. In imperdiet nisl a urna suscipit, ut aliquam nulla iaculis. Ut interdum augue nec ipsum semper imperdiet. Donec quis ligula ut arcu rhoncus maximus. Sed a metus blandit diam fringilla molestie id tempor turpis. Class aptent taciti sociosqu ad litora torquent per conubia nostra, per inceptos himenaeos. Vivamus nec lacus facilisis, dapibus eros vel, scelerisque est. Nunc sed scelerisque ligula, et convallis turpis.

\newpage
\thispagestyle{plain}
\begin{center}
    \large
    \textbf{Thesis Title}
    \vspace{0.4cm}
    \textbf{Pablo Santoro}
    \vspace{Abstract}
\end{center}
Lorem ipsum dolor sit amet, consectetur adipiscing elit. Cras at ullamcorper ligula. Fusce varius nibh dui, in semper lacus semper finibus. Donec ut orci eget nibh efficitur fringilla nec vel mi. Morbi ornare nunc vitae orci dictum, sed malesuada nisi accumsan. Curabitur ultrices vel urna sed tincidunt. Proin at cursus quam. Aenean pulvinar convallis vehicula. Mauris hendrerit ac ligula quis cursus. Sed at lorem vel sapien porttitor volutpat quis id libero.

Fusce porta est pellentesque, suscipit diam id, ultrices tellus. Nulla feugiat turpis erat, non mattis erat tincidunt non. Praesent vitae urna eu augue accumsan placerat non quis nulla. Mauris euismod dolor quis accumsan luctus. Etiam dolor mauris, consectetur pharetra tortor sed, dapibus scelerisque orci. Integer eu lectus sagittis, imperdiet ligula id, tempor justo. Etiam nec lacus ut erat vulputate hendrerit. Proin eleifend hendrerit vehicula. Etiam molestie ipsum eu metus hendrerit bibendum accumsan nec sapien. Proin mi dolor, lobortis ac ultrices ac, porta non sem. Nam vel nibh eget nunc cursus efficitur. Mauris eu elit lacinia, pulvinar tellus sit amet, accumsan leo. Nam venenatis arcu ipsum, ut rutrum dolor consectetur quis.

Nulla ac ex rutrum, faucibus tortor non, vulputate quam. Pellentesque tempor fringilla odio sed tincidunt. Donec condimentum felis sit amet erat posuere, ut blandit turpis congue. Praesent bibendum eget justo vitae ultrices. Vivamus aliquet tincidunt egestas. Ut at feugiat ante. Phasellus in nunc nec tellus venenatis lacinia. In imperdiet nisl a urna suscipit, ut aliquam nulla iaculis. Ut interdum augue nec ipsum semper imperdiet. Donec quis ligula ut arcu rhoncus maximus. Sed a metus blandit diam fringilla molestie id tempor turpis. Class aptent taciti sociosqu ad litora torquent per conubia nostra, per inceptos himenaeos. Vivamus nec lacus facilisis, dapibus eros vel, scelerisque est. Nunc sed scelerisque ligula, et convallis turpis.

\newpage

\section{Introducción}

Dentro de la ciencia de datos actual, una de las áreas que más se
están desarrollando en los últimos tiempos es la relacionada a las
arquitecturas de redes neuronales y, en particular, a aquellas relacionadas
con redes profundas, o ``Deep Learning'' en inglés.

Teóricamente, cualquier red neuronal de tres capas (entrada, oculta
y de salida) puede aproximar cualquier función, pero aún se desconoce
el entrenamiento apropiado para poder alcanzar esa meta teórica.
Por ende, se recurrió a agregar capas no redundantes que permiten
alcanzar mejores resultados a los que se estaban logrando a través
de las redes neuronales de tres capas, dando lugar a las llamadas
redes profundas.

Las redes profundas incidieron de manera decisiva en el procesamiento
de imágenes en dos y tres dimensiones. Se han desarrollado y se continúan
desarrollando redes con cada vez mejor precisión a la hora de reconocer
y reconstruir figuras de las más diversas. Para la reconstrucción
de modelos tridimensionales, las arquitecturas más comunes son los ``autoencoders'',
las llamadas ``Generative Adversarial Networks'' (GAN) y los ``variational
autoencoders'' (VAE). Recientemente se han comenzaron a desarrollar
arquitecturas que, además de poder reconstruir los ejemplos de entrenamiento,
puede generar nuevas imágenes, tomando nuevas muestras de datos de
un espacio continuo multivariado \cite{Karras2018}.

Un campo que está comenzando a tomar impulso recientemente es el de
procesamiento de datos en tres dimensiones usando redes generativas
con arquitecturas como GAN y VAE. Estas redes permiten, además de
reconstruir figuras con las que se entrenen, ajustar una distribución
multivariada (ajustada a tener una distribución normal estándar multivariada) a los fines de tener un espacio de muestreo continuo a través del cual poder generar nuevas figuras y poder hacer una interpolación continua en el espacio continuo de las figuras de entrenamiento. Esto permite poder explorar el espacio completo de las figuras de entrenamiento, con la posibilidad de encontrar figuras relacionadas a las originales pero con variaciones en los detalles, los cuales podrían ser útiles a los fines de entender y explorar el espacio de figuras con las que se trabajó en los datos originales.

\subsection{Convoluciones, pooling, normalización por lotes}

\section{Estado del arte}

\subsection{Variational autoencoders}

\subsection{Arquitecturas de Deep Learning en 3D}

En \cite{Gao2019} las redes aprenden un conjunto de voxeles deformables
para poder reconstruir figuras suaves, en lugar de reconstruir usando
voxeles. Su arquitectura consta de dos VAE: La primera, que llaman
``PartVAE'' se entrena un dataset con figuras etiquetadas por partes,
por lo que aprende a reconstruir las partes por separado. La segunda,
que llaman ``SP-VAE'' (por ``Structured Parts VAE'') aprende la
estructura global del objeto, asegurando luego la coherencia de las
partes ensambladas. En \cite{Muralikrishnan2019}, se utiliza la arquitectura VAE y la representación de objetos 3D en forma de voxeles, puntos y multiples vistas 2D para construir un mismo código latente que representa al mismo objeto, independientemente de su representación original y que permite reconstruir el mismo objeto en cualquiera de las tres formas de representación. Esta arquitectura se aplica tanto para tareas discriminativas como generativas. También existen otras formas derepresentaciones con performances destacables, como eluso de funciones de distancia con signo \cite{Park2019} (Signed Distance Functions, SDF, por sus siglas en inglés) en el que se muestrean puntos
cerca de la superficie de un objeto teniendo en cuenta si el punto
se encuentra dentro o fuera de la figura (representado por el signo)
y la distancia más corta del punto hacia la superficie del objeto
a entrenar. Esto permite que se calcule una SDF por cada una de las
figuras de entrenamiento, teniendo luego un auto encoder (o auto decoder, un tipo de arquitectura que no usa encoders para entrenar, como en el caso de \cite{Park2019}) que realizar una inferencia de las SDF
de cada figura y encontrar abstracciones entre las distintas SDF de
cada figura de los datos de entrenamiento. Hay nuevas arquitecturas
\cite{Olszewski2019} que también han logrado generar vistas nuevas
de objetos tridimensionales (Novel View Synthesis, o NVS por sus siglas
en inglés). En el caso de \cite{Olszewski2019}, se entrena usando varios objetos con múltiples vistas (imágenes en dos dimensiones) de un mismo objeto tridimensional en una arquitectura con encoder, un "bottleneck" (o "cuello de botella") y un decoder para recrear la imagen con las transformaciones aplicadas en el bottleneck. Lo interesante del método empleado consiste en que la NVS no es algo inferido por la red en el entrenamiento, sino que es aplicado de una forma no aprendida por la red en el bottleneck, lo que permite realizar otras transformaciones además de nuevas vistas de un objeto (como, por ejemplo, distintas poses de un ser humano) sin variar la forma de entrenar la red. Asimismo, las interpolaciones son diferenciables y permiten el correcto funcionamiento del algoritmo de backpropagation para poder actualizar los parámetros del encoder.

% Agregar figura de la arquitectura de la TBN de Olszewski et al.

\cite{Li2019} logra una mejor reconstrucción de figuras en tres dimensiones si se reconstruyen las partes por separado (teniendo datos de entrenamiento con partes claramente segmentadas, con el fin de poder dividirlas y usar distintos modelos para aprenderlas) y luego se ensamblan de acuerdo a la forma en que se corresponden las distintas partes, pudiendo tener leves variaciones respecto a las piezas originales pero correspondiéndose las partes entre sí a la hora del ensamble. Asimismo, eligiendo una de las piezas como ancla para el ensamble, pueden lograr generar distintas figuras dentro del espacio de figuras estudiado. La arquitectura que usan corresponde a un VAE-GAN: un variational autoencoder cuya reconstrucción de salida es la entrada de una red GAN. Esto genera un código latente formado por los códigos de las partes individuales, lo que alimenta una última red que es la encargada del ensamble de las partes de la figura.

% Agregar diagrama de la arquitectura usada por Li2019

\bibliographystyle{plain}
\addcontentsline{toc}{section}{\refname}
\bibliography{tesis_dm}

\end{document}
