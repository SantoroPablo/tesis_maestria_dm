\documentclass[12pt, spanish]{article}
\usepackage[utf8]{inputenc}
\usepackage[T1]{fontenc}
\usepackage{graphicx}
\usepackage{amsmath}
\usepackage{amssymb}
\usepackage[backend=biber,style=numeric]{biblatex}
\usepackage{float}
\usepackage{indentfirst}
\usepackage[a4paper, margin=1 in]{geometry}
\usepackage[spanish]{babel}
\usepackage{nameref}

%%%%%%%%%%%%%%%%%%%%%%%%%%%%%%%%%%%%%%%%%%%%%%%%
% Paquetes para usar graficos desde mathcha.io %
%%%%%%%%%%%%%%%%%%%%%%%%%%%%%%%%%%%%%%%%%%%%%%%%
\usepackage{tikz}
\usepackage{mathdots}
\usepackage{yhmath}
\usepackage{cancel}
\usepackage{color}
\usepackage{siunitx}
\usepackage{array}
\usepackage{multirow}
\usepackage{gensymb}
\usepackage{tabularx}
\usepackage{hyperref}
\usepackage{booktabs}
\usetikzlibrary{fadings}
\usetikzlibrary{calc}
\usetikzlibrary{patterns}
\usetikzlibrary{positioning}
\usetikzlibrary{shadows.blur}
\usetikzlibrary{shapes.geometric}
%%%%%%%%%%%%%%%%%%%%%%%%%%%%%%%%%%%%%%%%%%%%%%

%%%%%%%%%%%%%%%%%%%%%%%%%%%%%%%%%%%%%%%%%
%  Corrigiendo babel para uso con Tikz  %
%%%%%%%%%%%%%%%%%%%%%%%%%%%%%%%%%%%%%%%%%
\usetikzlibrary{babel,arrows,decorations,snakes,backgrounds}

%%%%%%%%%%%%%%%%%%%%%%%%%%%%%%%%%%%%%%%%%%%%%%%%%%%%%%
%  TikZ para graficar diagramas de redes neuronales  %
%%%%%%%%%%%%%%%%%%%%%%%%%%%%%%%%%%%%%%%%%%%%%%%%%%%%%%
\tikzstyle{startstop} = [rectangle, rounded corners,
    minimum width=1cm, minimum height=1cm,
    text centered, draw=black, fill=white,
    text width=3cm]
\tikzstyle{io} = [rectangle, rounded corners,
    minimum width=1cm, minimum height=1cm,
    text centered, draw=black, fill=white,
    text width=3cm]
\tikzstyle{arrow} = [->,>=stealth]

\addbibresource{tesis_dm.bib}
\graphicspath{ {./images/} }
\tikzset{every picture/.style={line width=0.75pt}} %set default line width to 0.75pt        

% Cambiando el interlineado
\linespread{1.3}

\begin{document}

\begin{titlepage}
    \begin{center}

    \huge
    \textbf{Tesis de Maestría en Explotación de Datos y Descubrimiento del Conocimiento}
    
    \vspace{0.5cm}
    
    \vfill
    \Large
    % Título tentativo
    Redes neuronales profundas para generación y manipulación de objetos tridimensionales.
    
    \vspace{0.5cm}
    
    \includegraphics[width=0.4\textwidth]{images/logo_uba.png}
    
    \vspace{0.5cm}
    \large
    Facultad de Ciencias Exactas y Naturales\\
    Maestría en Explotación de Datos y Descubrimiento del Conocimiento\\
    
    \vspace{0.5cm}
    
    \includegraphics[width=0.4\textwidth]{images/logo_maestria_dm.png}

    \vspace{0.5cm}
    
    Autor: Pablo Santoro
    
    \vspace{0.5cm}
    
    Director de Tesis:\\
    Dr. Emmanuel Iarussi
    
    {\selectlanguage{spanish}\today}
    
    \end{center}
\end{titlepage}

\thispagestyle{plain}
\begin{center}
    \large
    \textbf{Thesis Title}
    \vspace{0.4cm}
    \textbf{Pablo Santoro}
    \vspace{Resumen}
\end{center}
Lorem ipsum dolor sit amet, consectetur adipiscing elit. Cras at ullamcorper ligula. Fusce varius nibh dui, in semper lacus semper finibus. Donec ut orci eget nibh efficitur fringilla nec vel mi. Morbi ornare nunc vitae orci dictum, sed malesuada nisi accumsan. Curabitur ultrices vel urna sed tincidunt. Proin at cursus quam. Aenean pulvinar convallis vehicula. Mauris hendrerit ac ligula quis cursus. Sed at lorem vel sapien porttitor volutpat quis id libero.

Fusce porta est pellentesque, suscipit diam id, ultrices tellus. Nulla feugiat turpis erat, non mattis erat tincidunt non. Praesent vitae urna eu augue accumsan placerat non quis nulla. Mauris euismod dolor quis accumsan luctus. Etiam dolor mauris, consectetur pharetra tortor sed, dapibus scelerisque orci. Integer eu lectus sagittis, imperdiet ligula id, tempor justo. Etiam nec lacus ut erat vulputate hendrerit. Proin eleifend hendrerit vehicula. Etiam molestie ipsum eu metus hendrerit bibendum accumsan nec sapien. Proin mi dolor, lobortis ac ultrices ac, porta non sem. Nam vel nibh eget nunc cursus efficitur. Mauris eu elit lacinia, pulvinar tellus sit amet, accumsan leo. Nam venenatis arcu ipsum, ut rutrum dolor consectetur quis.

Nulla ac ex rutrum, faucibus tortor non, vulputate quam. Pellentesque tempor fringilla odio sed tincidunt. Donec condimentum felis sit amet erat posuere, ut blandit turpis congue. Praesent bibendum eget justo vitae ultrices. Vivamus aliquet tincidunt egestas. Ut at feugiat ante. Phasellus in nunc nec tellus venenatis lacinia. In imperdiet nisl a urna suscipit, ut aliquam nulla iaculis. Ut interdum augue nec ipsum semper imperdiet. Donec quis ligula ut arcu rhoncus maximus. Sed a metus blandit diam fringilla molestie id tempor turpis. Class aptent taciti sociosqu ad litora torquent per conubia nostra, per inceptos himenaeos. Vivamus nec lacus facilisis, dapibus eros vel, scelerisque est. Nunc sed scelerisque ligula, et convallis turpis.

\newpage
\thispagestyle{plain}
\begin{center}
    \large
    \textbf{Thesis Title}
    \vspace{0.4cm}
    \textbf{Pablo Santoro}
    \vspace{Abstract}
\end{center}
Lorem ipsum dolor sit amet, consectetur adipiscing elit. Cras at ullamcorper ligula. Fusce varius nibh dui, in semper lacus semper finibus. Donec ut orci eget nibh efficitur fringilla nec vel mi. Morbi ornare nunc vitae orci dictum, sed malesuada nisi accumsan. Curabitur ultrices vel urna sed tincidunt. Proin at cursus quam. Aenean pulvinar convallis vehicula. Mauris hendrerit ac ligula quis cursus. Sed at lorem vel sapien porttitor volutpat quis id libero.

Fusce porta est pellentesque, suscipit diam id, ultrices tellus. Nulla feugiat turpis erat, non mattis erat tincidunt non. Praesent vitae urna eu augue accumsan placerat non quis nulla. Mauris euismod dolor quis accumsan luctus. Etiam dolor mauris, consectetur pharetra tortor sed, dapibus scelerisque orci. Integer eu lectus sagittis, imperdiet ligula id, tempor justo. Etiam nec lacus ut erat vulputate hendrerit. Proin eleifend hendrerit vehicula. Etiam molestie ipsum eu metus hendrerit bibendum accumsan nec sapien. Proin mi dolor, lobortis ac ultrices ac, porta non sem. Nam vel nibh eget nunc cursus efficitur. Mauris eu elit lacinia, pulvinar tellus sit amet, accumsan leo. Nam venenatis arcu ipsum, ut rutrum dolor consectetur quis.

Nulla ac ex rutrum, faucibus tortor non, vulputate quam. Pellentesque tempor fringilla odio sed tincidunt. Donec condimentum felis sit amet erat posuere, ut blandit turpis congue. Praesent bibendum eget justo vitae ultrices. Vivamus aliquet tincidunt egestas. Ut at feugiat ante. Phasellus in nunc nec tellus venenatis lacinia. In imperdiet nisl a urna suscipit, ut aliquam nulla iaculis. Ut interdum augue nec ipsum semper imperdiet. Donec quis ligula ut arcu rhoncus maximus. Sed a metus blandit diam fringilla molestie id tempor turpis. Class aptent taciti sociosqu ad litora torquent per conubia nostra, per inceptos himenaeos. Vivamus nec lacus facilisis, dapibus eros vel, scelerisque est. Nunc sed scelerisque ligula, et convallis turpis.

\newpage

\section{Introducción}

Dentro de la ciencia de datos actual, una de las áreas que más se
han desarrollado en los últimos tiempos es la relacionada a las
arquitecturas de redes neuronales y, en particular, a aquellas relacionadas
con redes profundas, o ``Deep Learning'' en inglés.

Teóricamente, una arquitectura de tres capas (entrada, oculta
y de salida) puede aproximar cualquier función, pero aún se desconoce
el entrenamiento apropiado para poder alcanzar esa meta teórica.
Por ende, se recurrió a agregar capas no redundantes que permiten
alcanzar mejores resultados respecto a los lograndos a través
de las redes neuronales de tres capas, dando lugar a las redes
profundas.

Recientemente, las redes profundas comenzaron a incidir de manera decisiva en el procesamiento
de objetos en dos y tres dimensiones. Se continúan desarrollando arquitecturas 
con cada vez mejor precisión a la hora de reconocer
y reconstruir diversos objetos. Para la reconstrucción
de modelos tridimensionales, las arquitecturas básicas más comunes son los ``autoencoders'',
las llamadas ``Generative Adversarial Networks'' (GAN) y los ``Variational
AutoEncoders'' (VAE). Las arquitecturas VAE y GAN,
además de poder reconstruir los ejemplos de entrenamiento,
puede generar nuevas imágenes, tomando muestras de datos de
un espacio continuo multivariado \cite{Karras2018}. Además de
reconstruir las figuras de entrenamiento, estas arquitecturas permiten ajustar una distribución
multivariada (generalmente, una distribución normal estándar multivariada)
a los fines de tener un espacio de muestreo continuo a través del cual poder generar
nuevas figuras y poder hacer una interpolación continua en el espacio de
las figuras de entrenamiento. Esto permite poder explorar el espacio completo de
los objetos (a diferencia de los discontinuos espacios generados por los 
autoencoders tradicionales), con la posibilidad de encontrar variaciones en ciertos detalles
de los mismos, los cuales podrían ser útiles a los fines de entender y explorar el espacio
de figuras con las que se trabajó en los datos originales.

En este sentido, en este trabajo se muestra una variante de la arquitectura VAE, muy
aproximada a la arquitectura original pero con una variación en la forma de entrenamiento,
tomando la estrategia de códigos latentes del tipo \textit{overcomplete} \cite{Yin2019} para poder lograr no
sólo la reconstrucción de un dataset de figuras tridimensionales previamente segmentadas
en sus partes, sino también poder aislar las partes dentro del espacio de códigos latentes.
De esta forma, se logran aislar distintos detalles en los objetos en segmentos de los códigos
latentes.

\clearpage

\section{Estado del arte}
\label{sec:estado_del_arte}

\subsection{Redes neuronales} % y limitaciones y que viene a solucionar el Deep Learning
Las redes neuronales son un conjunto de arquitecturas diversas basadas todas ellas
en neuronas artificiales, las cuales representan una unidad de procesamiento de
información que recibe estímulos de entrada, los cuales se combinan de forma
lineal y a cuyo resultado se le aplica una función de activación, que da una respuesta
a la salida de la neurona. Existen muchas funciones de activación, como ser binarias
(valores 0 ó 1), lineales, acotadas entre dos valores cualesquiera, etc. Estas funciones
son las de más común uso en la práctica pero puede elegirse cualquier función
como activación de una neurona. La función de activación, no obstante, varía con el problema
que debe resolver la neurona: por ejemplo, si se tratan problemas de clasificación
se utilizan funciones de activación acotadas, normalmente entre $[0;1]$ o $[-1;1]$,
mientras que para los problemas de regresión se usan funciones cuyo dominio son
todos los números reales.
Asimismo, el resultado de la función de activación pueden ser valores continuos
o discretos, también funciones definidas por partes.
Los valores de activación de una neurona dependen sólo de los datos de entrada
y de los parámetros aprendidos $w$.

La representación tradicional de una neurona implica valores de entrada $x_i$,
cada uno de los cuales se multiplica por un peso $w_i$, sumándose luego cada uno de
los resultados de estos productos y se les aplica una función de activación ``g'',
que dan como resultado las salidas $\hat{y}$ \cite{isasi2004redes}.

\begin{figure}[H]
\centering
\tikzset{every picture/.style={line width=0.75pt}} %set default line width to 0.75pt        

\begin{tikzpicture}[x=0.75pt,y=0.75pt,yscale=-1,xscale=1]
%uncomment if require: \path (0,402); %set diagram left start at 0, and has height of 402

%Shape: Ellipse [id:dp2720015218944005] 
\draw   (110.38,315.73) .. controls (110.38,305.09) and (118.38,296.46) .. (128.27,296.46) .. controls (138.15,296.46) and (146.16,305.09) .. (146.16,315.73) .. controls (146.16,326.37) and (138.15,335) .. (128.27,335) .. controls (118.38,335) and (110.38,326.37) .. (110.38,315.73) -- cycle ;
%Shape: Ellipse [id:dp3837077404561182] 
\draw   (153.31,315.73) .. controls (153.31,305.09) and (161.32,296.46) .. (171.2,296.46) .. controls (181.08,296.46) and (189.09,305.09) .. (189.09,315.73) .. controls (189.09,326.37) and (181.08,335) .. (171.2,335) .. controls (161.32,335) and (153.31,326.37) .. (153.31,315.73) -- cycle ;
%Shape: Ellipse [id:dp01981874579517595] 
\draw   (196.25,315.73) .. controls (196.25,305.09) and (204.26,296.46) .. (214.14,296.46) .. controls (224.02,296.46) and (232.03,305.09) .. (232.03,315.73) .. controls (232.03,326.37) and (224.02,335) .. (214.14,335) .. controls (204.26,335) and (196.25,326.37) .. (196.25,315.73) -- cycle ;
%Shape: Ellipse [id:dp7236151148280978] 
\draw   (239.19,315.73) .. controls (239.19,305.09) and (247.2,296.46) .. (257.08,296.46) .. controls (266.96,296.46) and (274.97,305.09) .. (274.97,315.73) .. controls (274.97,326.37) and (266.96,335) .. (257.08,335) .. controls (247.2,335) and (239.19,326.37) .. (239.19,315.73) -- cycle ;
%Shape: Ellipse [id:dp7479569288356038] 
\draw   (24.5,200.12) .. controls (24.5,189.48) and (32.51,180.85) .. (42.39,180.85) .. controls (52.27,180.85) and (60.28,189.48) .. (60.28,200.12) .. controls (60.28,210.76) and (52.27,219.39) .. (42.39,219.39) .. controls (32.51,219.39) and (24.5,210.76) .. (24.5,200.12) -- cycle ;
%Shape: Ellipse [id:dp9939083257886165] 
\draw   (67.44,200.12) .. controls (67.44,189.48) and (75.45,180.85) .. (85.33,180.85) .. controls (95.21,180.85) and (103.22,189.48) .. (103.22,200.12) .. controls (103.22,210.76) and (95.21,219.39) .. (85.33,219.39) .. controls (75.45,219.39) and (67.44,210.76) .. (67.44,200.12) -- cycle ;
%Shape: Ellipse [id:dp4073834611348608] 
\draw   (289.28,200.12) .. controls (289.28,189.48) and (297.29,180.85) .. (307.17,180.85) .. controls (317.05,180.85) and (325.06,189.48) .. (325.06,200.12) .. controls (325.06,210.76) and (317.05,219.39) .. (307.17,219.39) .. controls (297.29,219.39) and (289.28,210.76) .. (289.28,200.12) -- cycle ;
%Shape: Ellipse [id:dp22167190095519107] 
\draw   (332.22,200.12) .. controls (332.22,189.48) and (340.23,180.85) .. (350.11,180.85) .. controls (359.99,180.85) and (368,189.48) .. (368,200.12) .. controls (368,210.76) and (359.99,219.39) .. (350.11,219.39) .. controls (340.23,219.39) and (332.22,210.76) .. (332.22,200.12) -- cycle ;
%Straight Lines [id:da15450774738183992] 
\draw    (42.39,219.39) -- (128.27,296.46) ;
%Straight Lines [id:da2829044989577385] 
\draw    (42.39,219.39) -- (171.2,296.46) ;
%Straight Lines [id:da8980563157039807] 
\draw    (42.39,219.39) -- (214.14,296.46) ;
%Straight Lines [id:da26277993551748025] 
\draw    (42.39,219.39) -- (257.08,296.46) ;
%Straight Lines [id:da5528352589902739] 
\draw    (85.33,219.39) -- (128.27,296.46) ;
%Straight Lines [id:da10791289044527552] 
\draw    (85.33,219.39) -- (171.2,296.46) ;
%Straight Lines [id:da9310393684628768] 
\draw    (85.33,219.39) -- (214.14,296.46) ;
%Straight Lines [id:da48581876734991836] 
\draw    (85.33,219.39) -- (257.08,296.46) ;
%Straight Lines [id:da011444021643488345] 
\draw    (307.17,219.39) -- (128.27,296.46) ;
%Straight Lines [id:da4811843544333507] 
\draw    (307.17,219.39) -- (171.2,296.46) ;
%Straight Lines [id:da521913029486315] 
\draw    (307.17,219.39) -- (214.14,296.46) ;
%Straight Lines [id:da7939068226488193] 
\draw    (307.17,219.39) -- (257.08,296.46) ;
%Straight Lines [id:da15329924550198482] 
\draw    (350.11,219.39) -- (128.27,296.46) ;
%Straight Lines [id:da19526208362823816] 
\draw    (344.03,221.7) -- (171.2,296.46) ;
%Straight Lines [id:da9795575332823692] 
\draw    (350.11,219.39) -- (214.14,296.46) ;
%Straight Lines [id:da043010971324199065] 
\draw    (350.11,219.39) -- (257.08,296.46) ;
%Shape: Ellipse [id:dp6288164009507364] 
\draw   (278.43,84.21) .. controls (278.46,94.85) and (270.48,103.5) .. (260.6,103.54) .. controls (250.72,103.57) and (242.68,94.97) .. (242.65,84.33) .. controls (242.61,73.69) and (250.6,65.04) .. (260.48,65) .. controls (270.36,64.97) and (278.39,73.57) .. (278.43,84.21) -- cycle ;
%Shape: Ellipse [id:dp9685549887103893] 
\draw   (235.49,84.36) .. controls (235.52,95) and (227.54,103.65) .. (217.66,103.69) .. controls (207.78,103.72) and (199.74,95.12) .. (199.71,84.48) .. controls (199.68,73.84) and (207.66,65.19) .. (217.54,65.15) .. controls (227.42,65.12) and (235.46,73.72) .. (235.49,84.36) -- cycle ;
%Shape: Ellipse [id:dp5999333628500731] 
\draw   (192.55,84.51) .. controls (192.59,95.15) and (184.6,103.81) .. (174.72,103.84) .. controls (164.84,103.88) and (156.81,95.28) .. (156.77,84.64) .. controls (156.74,74) and (164.72,65.34) .. (174.6,65.31) .. controls (184.48,65.27) and (192.52,73.87) .. (192.55,84.51) -- cycle ;
%Shape: Ellipse [id:dp4986242138618515] 
\draw   (149.61,84.67) .. controls (149.65,95.31) and (141.67,103.96) .. (131.79,104) .. controls (121.91,104.03) and (113.87,95.43) .. (113.84,84.79) .. controls (113.8,74.15) and (121.78,65.5) .. (131.66,65.46) .. controls (141.54,65.43) and (149.58,74.03) .. (149.61,84.67) -- cycle ;
%Straight Lines [id:da964227445843808] 
\draw    (346.71,180.3) -- (260.6,103.54) ;
%Straight Lines [id:da9317281730322] 
\draw    (346.71,180.3) -- (217.66,103.69) ;
%Straight Lines [id:da8781372263812648] 
\draw    (346.71,180.3) -- (174.72,103.84) ;
%Straight Lines [id:da7984922350871231] 
\draw    (346.71,180.3) -- (131.78,104) ;
%Straight Lines [id:da9933018474502304] 
\draw    (303.77,180.46) -- (260.6,103.54) ;
%Straight Lines [id:da8069788202243142] 
\draw    (303.77,180.46) -- (217.66,103.69) ;
%Straight Lines [id:da07714153983529748] 
\draw    (303.77,180.46) -- (174.72,103.84) ;
%Straight Lines [id:da7233505714762876] 
\draw    (303.77,180.46) -- (131.78,104) ;
%Straight Lines [id:da7186337846401032] 
\draw    (81.93,181.25) -- (260.6,103.54) ;
%Straight Lines [id:da3437340674773921] 
\draw    (81.93,181.25) -- (217.66,103.69) ;
%Straight Lines [id:da31076367922801573] 
\draw    (81.93,181.25) -- (174.72,103.84) ;
%Straight Lines [id:da08571577701097399] 
\draw    (81.93,181.25) -- (131.78,104) ;
%Straight Lines [id:da1918176912687859] 
\draw    (38.99,181.4) -- (260.6,103.54) ;
%Straight Lines [id:da13026881483313502] 
\draw    (45.07,179.07) -- (217.66,103.69) ;
%Straight Lines [id:da14565851774081717] 
\draw    (38.99,181.4) -- (174.72,103.84) ;
%Straight Lines [id:da17398502333326027] 
\draw    (38.99,181.4) -- (131.78,104) ;
%Straight Lines [id:da35143564995955634] 
\draw    (128.27,335) -- (128.5,369) ;
%Straight Lines [id:da2874884733697052] 
\draw    (214.14,335) -- (214.38,369) ;
%Straight Lines [id:da5497244110106112] 
\draw    (171.2,335) -- (171.44,369) ;
%Straight Lines [id:da6991413964424229] 
\draw    (257.08,335) -- (257.31,369) ;
%Straight Lines [id:da6414673286784804] 
\draw    (131.43,31.46) -- (131.66,65.46) ;
%Straight Lines [id:da8676044351416776] 
\draw    (217.31,31.15) -- (217.54,65.15) ;
%Straight Lines [id:da8848276849172192] 
\draw    (174.37,31.31) -- (174.6,65.31) ;
%Straight Lines [id:da5213836105071521] 
\draw    (260.24,31) -- (260.48,65) ;

% Text Node
\draw (160,373) node [anchor=north west][inner sep=0.75pt]   [align=left] {Entradas};
% Text Node
\draw (171,8) node [anchor=north west][inner sep=0.75pt]   [align=left] {Salidas};
% Text Node
\draw (139,173) node [anchor=north west][inner sep=0.75pt]  [font=\huge] [align=left] {...};
% Text Node
\draw (224,174) node [anchor=north west][inner sep=0.75pt]  [font=\huge] [align=left] {...};

\end{tikzpicture}
\caption{Esquema de una red neuronal ordenada por capas. De abajo hacia arriba:
capa de entrada, oculta y de salida. Los círculos representan las neuronas, las
líneas representan las conexiones entre las capas. Este caso representado es de
una red completamente conectada, en donde una neurona se conecta con todas las
de la capa siguiente. El flujo de información es desde abajo hacia arriba.}
\label{red_fully_connected}
\end{figure}

Las redes neuronales por ende, son modelos de neuronas interconectadas, como
se puede observar en el modelo representado en la figura \ref{red_fully_connected}.
La arquitectura utilizada comúnmente consta de distintas capas interconectadas
entre sí: una capa de entrada, desde la cual se toman los datos y cuya salida
es tomada por una o más capas ocultas. Finalmente, se encuentra una capa de
salida, tras la cual se obtiene un valor de salida de la red. Todas las capas
reciben una entrada de cada una de las neuronas de la capa anterior, y todas
las variables de los datos procesados se reciben en la capa de entrada.
Este tipo de arquitecturas es el más común para problemas de clasificación o de
regresión, y generalmente constan de una sola capa oculta.
La cantidad de neuronas en cada capa es un \textit{hiperparámetro} del modelo, que
se debe fijar antes de entrenar el modelo.
% TODO: revisar los comentarios de Emmaniel sobre la siguiente sección que separé
%       entre comentarios: no suma mucho y posiblemente se pueda resumir en un par
%       de oraciones
Ciertas convenciones se toman, sin embargo, para las capas de entrada y salida:
\begin{itemize}
    \item Para la capa de entrada, tantas neuronas como variables tengan los
        datos que se procesen.
    \item Para la capa de salida:
    \begin{itemize}
        \item Si es una red de clasificación, una neurona para cada uno de los
        resultados posibles en la red. Por ejemplo, para una clasificación binaria
        (1 ó 0), se utilizarían dos neuronas para cada uno de los estados posibles.
        \item En el caso de una regresión se utiliza una neurona con activación lineal
        (que da como salida un valor continuo real).
    \end{itemize}
\end{itemize}
Dadas estas convenciones,
% TODO: FIN DE SECCIÓN
El hiperparámetro fundamental a resolver es la cantidad de neuronas necesarias en
la capa oculta. La forma de determinarlo es, generalmente, de forma empírica, de
acuerdo al error resultante del modelado de
acuerdo a la arquitectura determinada. Por otro lado, no se usan funciones lineales
en las capas ocultas, pues se puede determinar analíticamente que usar este tipo de
funciones para las capas ocultas es trivial, ya que se podrían reemplazar por una
sola capa que reúna las características de este conjunto \cite{isasi2004redes}. 
Esta arquitectura, de forma teórica, se considera un aproximador universal de
cualquier función matemática posible \cite{Hornik_1989}. Si bien no existe un
método universal para entrenar este tipo de redes para aproximar cualquier función,
sí se ha conseguido incrementar el desempeño de estos modelos desarrollando mejores 
funciones de activación que posibiliten incrementar la cantidad de capas ocultas.
En general, esto ha resultado en redes más precisas, y cuyo entrenamiento en general
no recae en el problema de gradientes que se aproximan a cero (propio de las funciones continuas acotadas,
cuya derivada está acotada en el intervalo $[0;1]$) \cite{Bengio_1994}, con
muchas más capas y cantidad de neuronas.
Este tipo de redes, que concatenan un número considerable de capas ocultas
son las que conforman el conjunto de técnicas que denominamos ``Deep Learning''
(Aprendizaje Profundo). Una de las funciones de activación más comunes en este
campo es la función ReL (\textit{Rectified Linear}, por sus siglas en inglés),
que toma el valor 0 para los valores de entrada negativos y el mismo valor de
entrada si el valor es positivo o cero. Funciona como si fuera una compuerta
que no permite valores negativos o cero a la entrada y propaga valores positivos
de la misma a la siguiente capa. Las neuronas que llevan como función de activación
esta función se llaman ``ReLU''. Esta función tiene las
propiedades interesantes de no acotar la activación producida por los datos de
entrada en la neurona y tiene una derivada constante igual a 1, por lo que no cae en el
problema de gradientes cada vez más chicos, lo que permite el concatenado profundo
de capas y que todas ellas puedan adaptarse más rápido a los errores, lo que
se traduce también en un entrenamiento más rápido del modelo. Para obtener
resultados acotados se siguen usando las tradicionales funciones sigmoideas y de
tangente hiperbólica, con propiedades interesantes en cuanto a las derivadas de
estas funciones, que facilitaron la programación de redes neuronales, a la vez
que tienen la propiedad matemática de acotar los valores de activación y ser
funciones continuas y derivables en todo su dominio.

\subsubsection{Entrenamiento de las redes neuronales}
El entrenamiento de una red neuronal es iterativo y consiste en la optimización
de una función de pérdida. De acuerdo al problema tratado, esta función de
pérdida toma distintas formas. Para el caso de la reconstrucción de objetos,
como es el caso de los autoencoders y los VAE, la función de pérdida toma la
siguiente forma:


\[
J(\theta)=E_{(x)\sim p_{data}}[L(f(x;\theta),x)]
\]

Donde $x$ representa los datos de entrenamiento, $L$ es la función
de pérdida, $f(x;\theta)$ es la salida de la red neuronal y $\hat{p}_{data}$
es la distribución de los datos. La red, en tiempo de entrenamiento,
tiene como modelo de optimización el minimizar $J(\theta)$ aproximado
mediante un estadístico insesgado. Entonces, esta función de pérdida
empírica toma la forma:

\[
J^{*}(\theta)=\frac{1}{m}\sum_{i=1}^{m}L(f(x^{(i)};\theta),x^{(i)})
\]

Donde $m$ es el número de ejemplos de entrenamiento, $x$ representando
el conjunto de datos de entrenamiento.

La red ``aprende'' de los errores cometidos en la reconstrucción de objetos
(valor de la salida de la red, $f(x;\theta)$). Para aprender, la red adapta
los valores de los pesos $w_i$ de acuerdo al signo y la magnitud del error
de predicción.  La forma de actualizar todos los $w_i$ de la red en las
distintas capas que la componen es mediante el algoritmo de propagación hacia
atrás (más conocido por su locución inglesa, ``backpropagation''). Este
algoritmo tiene en cuenta el gradiente del error a la hora de propagar el error
hacia todas las neuronas de la red. Asimismo, para evitar cambios de magnitud
considerable se ajusta cada uno de los cambios en una proporción de igual
magnitud para toda la red y prefijada de antemano (por lo que constituye otro
hiperparámetro para la red): la tasa de aprendizaje, o ``learning rate''. Esta
tasa permite un aprendizaje más lento e incremental e impide que la red produzca
ajustes que le impidan encontrar valores mínimos de la función de error. Ahora
bien, como probablemente nunca se llegue a eliminar completamente el error de
entrenamiento en un problema no trivial, la red buscará constantemente eliminar
el mismo, por lo que el entrenamiento nunca finalizaría. A este problema se lo
puede encarar de dos formas distintas: prefijando una cantidad de iteraciones a
realizar en tiempo de entrenamiento o un límite de error aceptable tras el cual,
una vez alcanzados valores debajo de este umbral, se detiene el entrenamiento.
Generalmente, se utilizan las dos condiciones y se finaliza el entrenamiento
cuando se alcance alguna de las dos condiciones primero.

Desafortunadamente, debido a que, a priori, la superficie multidimensional
descrita por la función de error $L(f(x,\theta),x)$ es desconocida, varía con el problema
considerado y la forma de entrenamiento es iterativa, no se puede
garantizar que el mínimo que se encuentra tras el modelado sea el mínimo
absoluto de la función de error. Pueden probarse distintos valores de ``learning
rate'' para evitar que el modelo caiga en mínimos locales. También se suele
utilizar un coeficiente de ``inercia'', que sirve para actualizar las $w_i$
dependiendo de valores anteriores para dichos pesos. Aún así, el modelado
depende de todos estos hiperparámetros y se pueden realizar varias rondas de
entrenamiento para poder encontrar la configuración deseada de forma empírica.

% TODO: NUEVA SECCIÓN ESCRITA PARA EL ENTRENAMIENTO DE LAS REDES NEURONALES PROFUNDAS

El entrenamiento de las redes neuronales profundas es muy costoso
en tiempo computacional. Para evitar este problema y reducir tiempos,
se han diseñado técnicas de optimización para dicho entrenamiento.
Para este caso, se desea conocer un conjunto de parámetros $\theta$
de una red neuronal profunda de manera que minimice una función de
costos $J(\theta)$ relacionada con el desempeño de la red neuronal
en entrenamiento. De esta forma, para nuestro caso, en donde buscamos
minimizar la diferencia entre los objetos generados y los objetos
utilizados en entrenamiento, la función a minimizar queda expresada:

\[
J(\theta)=E_{(x)\sim p_{data}}[L(f(x;\theta),x)]
\]

Donde $x$ representa los datos de entrenamiento, $L$ es la función
de pérdida, $f(x;\theta)$ es la salida de la red neuronal y $\hat{p}_{data}$
es la distribución de los datos. La red, en tiempo de entrenamiento,
tiene como modelo de optimización el minimizar $J(\theta)$ aproximado
mediante un estadístico insesgado. Entonces, esta función de pérdida
empírica toma la forma:

\[
J^{*}(\theta)=\frac{1}{m}\sum_{i=1}^{m}L(f(x^{(i)};\theta),x^{(i)})
\]

Donde $m$ es el número de ejemplos de entrenamiento, $x$ representando
el conjunto de datos de entrenamiento.

Ahora bien, la capacidad de adaptación que tienen las redes neuronales
profundas hacen que sean propensas a caer en un sobreajuste (ya que poseen
una cantidad de parámetros tal que podrían memorizar directamente los datos
de entrenamiento). Varias de las técnicas de optimización actuales corrigen
este problema para evitar el sobreajuste y, por ende, obtener un mayor grado
de generalización.

% TODO: FIN DE LA SECCIÓN

\subsection{Convoluciones, pooling, normalización por lotes}

% En la versión de 2017 del libro "Deep Learning" de Goodfellow, Bengio y Courville
% a partir de la página 330 comienza el capítulo de redes convolucionales.
Las redes convolucionales son redes neuronales en las que al menos una de sus capas
usan una operación de convolución en lugar de una multiplicación matricial
\cite{goodfellow2016deep}. Este tipo de capas es fundamental para el procesamiento
de tipos de datos que presentan un formato de cuadrícula, como las imágenes
bidimensionales o los objetos tridimensionales. Es por ello que son uno de los
bloques fundamentales para las redes que tratan con este tipo de datos.

La convolución es una operación matemática que consiste en una media ponderada sobre
los datos de entrada (en el caso de \(t\) discreto):

\[s(t) = (x \circledast w)(t) = \sum_{a = -\infty}^{\infty} x(a)w(t-a) \]

Donde \(x\) representa los datos de entrada, \(w\) son los pesos del promedio
ponderado, que también se denominan \textit{kernel}.
Para el caso discreto bidimensional discreto (imágenes, por ejemplo), en lugar de
una suma infinita sólo se realiza sobre la suma de los valores de las matrices de
entrada a la convolución. También el kernel \(w\) toma la forma de una matriz
bidimensional con pesos adaptables por la red. Por ende, en el caso de una imagen
\(I\) con un kernel \(K\):

\[S(i,j)=(I \circledast K)(i,j)=\sum_{m}^{n} I(m,n)K(i-m,j-n)\]

Una particularidad de las capas convolucionales, que reduce muchísimo la
complejidad computacional del entrenamiento de una red profunda, son las conexiones
escasas que tienen las neuronas, asociadas al kernel convolucional, a diferencia de
las redes neuronales tradicionales en las que todas las neuronas se encuentran
conectadas con todas las de la capa siguiente. Esta reducción en el nivel de
conexiones permite, además, un aprendizaje local y una reducción en los grados de
libertad del modelo estadístio, lo que le ha permitido tener un mejor nivel de
generalización respecto a redes con capas completamente conectadas.

A continuación se destaca un ejemplo de aplicar la operación de convolución a una
matriz de entrada bidimensional:

\begin{figure}[H]
\centering
\begin{tikzpicture}[x=0.75pt,y=0.75pt,yscale=-1,xscale=1]
%uncomment if require: \path (0,460); %set diagram left start at 0, and has height of 460

\draw   (115,33) -- (165,33) -- (165,83) -- (115,83) -- cycle ; %Shape: Square [id:dp008348541907688967] 
\draw   (177,33) -- (227,33) -- (227,83) -- (177,83) -- cycle ; %Shape: Square [id:dp5695495374345101] 
\draw   (238,33) -- (288,33) -- (288,83) -- (238,83) -- cycle ; %Shape: Square [id:dp08913217477595681] 
\draw   (300,33) -- (350,33) -- (350,83) -- (300,83) -- cycle ; %Shape: Square [id:dp09565678498451557] 
\draw   (115,93) -- (165,93) -- (165,143) -- (115,143) -- cycle ; %Shape: Square [id:dp6616061503281512] 
\draw   (177,93) -- (227,93) -- (227,143) -- (177,143) -- cycle ; %Shape: Square [id:dp9731322863310219] 
\draw   (238,93) -- (288,93) -- (288,143) -- (238,143) -- cycle ; %Shape: Square [id:dp1734488588540104] 
\draw   (300,93) -- (350,93) -- (350,143) -- (300,143) -- cycle ; %Shape: Square [id:dp783784734771654] 
\draw   (115,153) -- (165,153) -- (165,203) -- (115,203) -- cycle ; %Shape: Square [id:dp3810445646111744] 
\draw   (177,153) -- (227,153) -- (227,203) -- (177,203) -- cycle ; %Shape: Square [id:dp597477459287147] 
\draw   (238,153) -- (288,153) -- (288,203) -- (238,203) -- cycle ; %Shape: Square [id:dp3259061174876716] 
\draw   (300,153) -- (350,153) -- (350,203) -- (300,203) -- cycle ; %Shape: Square [id:dp9728278094881708] 
\draw   (112,29) -- (230,29) -- (230,147) -- (112,147) -- cycle ; %Shape: Square [id:dp8155679517421655] 
\draw   (399,60) -- (449,60) -- (449,110) -- (399,110) -- cycle ; %Shape: Square [id:dp394644813826424] 
\draw   (461,60) -- (511,60) -- (511,110) -- (461,110) -- cycle ; %Shape: Square [id:dp8614370121086297] 
\draw   (399,120) -- (449,120) -- (449,170) -- (399,170) -- cycle ; %Shape: Square [id:dp595415369573254] 
\draw   (461,120) -- (511,120) -- (511,170) -- (461,170) -- cycle ; %Shape: Square [id:dp8309992031835938] 
\draw   (396,56) -- (514,56) -- (514,174) -- (396,174) -- cycle ; %Shape: Square [id:dp77375541391222] 
\draw   (117,233) -- (213.5,233) -- (213.5,329.5) -- (117,329.5) -- cycle ; %Shape: Square [id:dp2906572996014607] 
\draw   (111,228) -- (219.5,228) -- (219.5,336.5) -- (111,336.5) -- cycle ; %Shape: Square [id:dp4766592365283635] 
\draw   (224,234) -- (320.5,234) -- (320.5,330.5) -- (224,330.5) -- cycle ; %Shape: Square [id:dp690329106444943] 
\draw   (330,234) -- (426.5,234) -- (426.5,330.5) -- (330,330.5) -- cycle ; %Shape: Square [id:dp14628491601258453] 
\draw   (117,344) -- (213.5,344) -- (213.5,440.5) -- (117,440.5) -- cycle ; %Shape: Square [id:dp268759623182556] 
\draw   (224,345) -- (320.5,345) -- (320.5,441.5) -- (224,441.5) -- cycle ; %Shape: Square [id:dp5482508582899983] 
\draw   (330,345) -- (426.5,345) -- (426.5,441.5) -- (330,441.5) -- cycle ; %Shape: Right Angle [id:dp2438723945511756] 
\draw   (62.09,279.98) -- (62.09,86.11) -- (111.33,86.11) ; %Straight Lines [id:da4032615230182004] 
\draw   (62.09,279.98) -- (109.33,279.98) ;
\draw [shift={(111.33,279.98)}, rotate = 180] [color={rgb, 255:red, 0; green, 0; blue, 0}][line width=0.75] (10.93,-3.29) .. controls (6.95,-1.4) and (3.31,-0.3) .. (0,0) .. controls (3.31,0.3) and (6.95,1.4) .. (10.93,3.29);


\draw (140,58) node   [align=left] {$\displaystyle a$}; % Text Node
\draw (202,58) node   [align=left] {$\displaystyle b$}; % Text Node
\draw (264,58) node   [align=left] {$\displaystyle c$}; % Text Node
\draw (326,58) node   [align=left] {$\displaystyle d$}; % Text Node
\draw (140,118) node   [align=left] {$\displaystyle e$}; % Text Node
\draw (202,118) node   [align=left] {$\displaystyle f$}; % Text Node
\draw (264,118) node   [align=left] {$\displaystyle g$}; % Text Node
\draw (326,118) node   [align=left] {$\displaystyle h$}; % Text Node
\draw (140,178) node   [align=left] {$\displaystyle i$}; % Text Node
\draw (202,178) node   [align=left] {$\displaystyle j$}; % Text Node
\draw (264,178) node   [align=left] {$\displaystyle k$}; % Text Node
\draw (326,178) node   [align=left] {$\displaystyle l$}; % Text Node
\draw (129,16) node   [align=left] {Input}; % Text Node
\draw (424,85) node   [align=left] {$\displaystyle w$}; % Text Node
\draw (486,85) node   [align=left] {$\displaystyle x$}; % Text Node
\draw (424,145) node   [align=left] {$\displaystyle y$}; % Text Node
\draw (486,145) node   [align=left] {$\displaystyle z$}; % Text Node
\draw (418,43) node   [align=left] {Kernel}; % Text Node
% Text Node
\draw (169,284) node   [align=left] {$\displaystyle
\begin{array}{{>{\displaystyle}l}}
aw\ +bx\ +\ \\
ey\ \ +\ fz
\end{array}$};
% Text Node
\draw (276,285) node   [align=left] {$\displaystyle
\begin{array}{{>{\displaystyle}l}}
bw\ +cx\ +\ \\
fy\ \ +\ gz
\end{array}$};
% Text Node
\draw (382,286) node   [align=left] {$\displaystyle
\begin{array}{{>{\displaystyle}l}}
cw\ +dx\ +\ \\
gy\ \ +\ hz
\end{array}$};
% Text Node
\draw (168,393) node   [align=left] {$\displaystyle
\begin{array}{{>{\displaystyle}l}}
ew\ +fx\ +\ \\
iy\ \ +\ jz
\end{array}$};
% Text Node
\draw (275,394) node   [align=left] {$\displaystyle
\begin{array}{{>{\displaystyle}l}}
fw\ +gx\ +\ \\
jy\ \ +\ kz
\end{array}$};
% Text Node
\draw (381,395) node   [align=left] {$\displaystyle
\begin{array}{{>{\displaystyle}l}}
gw\ +hx\ +\ \\
ky\ \ +\ lz
\end{array}$};

\end{tikzpicture}
\caption{Un ejemplo de convolución bidimensional \cite{goodfellow2016deep}}
\label{convolucion}
\end{figure}

A la salida de una capa convolucional generalmente hay una capa de neuronas
ReLU detectoras, tras la cual puede habitualmente encontrarse una capa de
pooling, que consiste en tomar operaciones estándar a los resultados de los
kernels. Por ejemplo, de los resultados obtenidos por la convolución, se
pueden tomar ciertos estadísticos como el máximo o el promedio. El efecto
de esta capa es permitir una cierta invarianza a transformaciones en el
objeto de entrada, sobre todo a los desplazamientos. Es decir, en ciertos
modelos es preferible detectar ciertos patrones, como bordes, en cualquier
parte de la figura y no solo en algunos lugares. En otros modelos, con el
debido preprocesamiento de datos, puede no desearse este efecto y sí captar
características locales. No obstante, las operaciones de pooling suelen ser
bastante comunes en diversas arquitecturas.
Estas tres capas, convolucional, ReLU y pooling, habitualmente son referidas
como una sola capa convolucional, debido a lo común de este esquema.
%%%%%%%%%%%%%%%%%%%%%%%%%
%  Batch Normalization  %
%%%%%%%%%%%%%%%%%%%%%%%%%
En el entrenamiento, ademas de poder utilizar estas capas, también se puede
usar, para apovechar de mejor manera las capacidades de la GPU, la técnica
conocida como normalización por lotes (textit{Batch Normalization} en inglés),
que consiste en promediar los gradientes de varias pasadas de datos de
entrenamiento en la red, en lugar de actualizar los pesos cada vez que se pasa
un nuevo dato por la red. 
%%%%%%%%%%%%%%%%%%%%%%%
%  ADAM optimization  %
%%%%%%%%%%%%%%%%%%%%%%%
Asimismo, para mejorar el desempeño del modelos utilizamos la optimización
ADAM \cite{arsalan2017synthesizing}.

% Hasta acá, un resumen con otras palabras del libro de Goodfellow
% TODO: Revisar si conviene agregar información sobre las sigueintes
%       operaciones que se realizan cuando hablamos de deep learning,
%       que van a servir para no tener que volver a explicar de nuevo
%       cuando exponga mi arquitectura.
%           * Argumentos para los efectos de distintos Beta para la
%             KL Divergence
%       También tener en cuenta de explicar lo que sea necesario sobre
%       el batch size usado, no sólo para poder correr el modelo sino
%       en términos de la performance también.

\clearpage

\subsection{Arquitecturas de Deep Learning} % TODO: Nombraría a esta sección como 
% "Arquitecturas de deep learning" y nombraría a los autoencoders, a los
% GAN (incluyendo también la mejora, WGAN) y algunos ejemplos del éxito que
% han logrado en dos dimensiones.
Las arquitecturas de deep learning más modernas, que hacen uso del tipo
de capas convolucionales vistas en el apartado anterior, son las siguientes:
\begin{enumerate}
    \item Autoencoders (AE) \cite[p.~517]{goodfellow2016deep} y
        Variational autoencoders (VAE) \cite{Kingma2013}
    \item Redes Generativas Antagónicas (GAN \textit{Generative
        Adversarial Networks}, por sus siglas en inglés)
\end{enumerate}
Los AE son arquitecturas enfocadas en que su salida sea igual a su
entrada. Consta de dos partes, un \textit{encoder}, que toma los datos
de entrada y, en sucesivas capas del mismo, disminuye la dimensionalidad
de los datos y los traduce en un \textit{código latente}: un vector que
representa al dato de una forma más comprimida. La segundo parte, el
\textit{decoder}, toma el código latente e intenta reconstruir los datos
originales. La optimización se da midiendo el error de reconstrucción
a la salida del autoencoder. Muchas aplicaciones de AE resultan en un
código latente de una dimensionalidad menor a los datos, llamados
\textit{undercomplete autoencoders}, lo que resulta en una compresión
de los datos de entrada, generalmente con pérdida de información. 
Los VAE son una variante de los autoencoders tradicionales, del cual
modifican la forma en que la arquitectura representa los datos de 
entrada en el espacio de códigos latentes.  En los AE, el \textit{encoder}
representa los objetos en un punto en el espacio descrito por los códigos
latentes, mientras que en los VAE estos se representan como una
distribución \cite{Foster2019}. La distribución a priori de los datos de
entrada es desconocida, pero generalmente se asume una distribución normal
multivariada en el espacio de códigos latentes, utilizando la divergencia
de Kullback-Leibler en la función de pérdida para que la distribución del
espacio de los códigos latentes tienda a la distribución a priori.
La ventaja de esto es que brinda continuidad, al no ser ya un punto
discontinuo de otros objetos sino teniendo en cuenta que el espacio
representa la probabilidad de ser un objeto u otro, permitiendo
transiciones más suaves en el espacio de códigos latentes de un objeto a
otro de una forma continua, mientras que en el espacio formado por los
AE no se asume ninguna forma distribucional y el espacio es sumamente
discontinuo, no permitiendo la interpolación entre códigos latentes como
en los VAE. Debido a esta propiedad y al asumir una forma distribucional
una vez entrenada la red se pueden hacer muestreos sobre el espacio de
códigos latentes y generar nuevos objetos usando sólo el \textit{decoder},
figuras que nunca se usaron para entrenar. La interpolación, además de
ser generativa, permite observar características intermedias entre objetos
que en los datos de entrenamiento no se observan.
Por otra parte, lo que se espera de los AE y VAE, en lugar de una perfecta reconstrucción de
los datos de entrada, es una aplicación del código latente que resulte
útil, generalmente usado para encontrar variables latentes en los datos
de entrenamiento, que resultan en una generalización de la red respecto
a lo que observa durante el entrenamiento. Esto tiene sentido sobre todo
en los AE que utilizan capas convolucionales, las cuales representan
filtros aprendidos aplicados de forma sucesiva, los cuales encuentran
variables latentes en las imágenes o figura. Por ejemplo, una arquitectura
que sea entrenada con imágenes para reconocer rostros \cite{Higgins2017}
puede encontrar variables latentes complejas como ser los distintos
rasgos faciales. Al usar la red entrenada para generar nuevas imágenes,
la manipulacióń de los códigos latentes encontrados sirve para alterar
estas variables latentes encontradas por los autoencoders. No obstante,
la manipulación de una de las componentes de un vector de código
no conlleva, generalmente, la manipulación de una de las variables latentes
encontrada por el autoencoder, sin afectar otras variables. Es decir,
alterar una componente de un vector de código latente correspondiente
a un objeto puede alterar completamente la salida por el \textit{decoder}.
En esta situación, las variables latentes se encuentran \textit{enredadas}
dentro de las componentes del código latente, con una alta correlación entre
sí. Este efecto (llamado \textit{entangling} en inglés) dio lugar a varias
estrategias para lograr \textit{desenredar} las componentes de los códigos
latentes. Diversas estrategias se emplean tanto a nivel de modificaciones
a la función de pérdida \cite{Higgins2017} como a distintas arquitecturas
% TODO: Quizá estaría bueno tener en cuenta más papers en esta parte, en
%       especial de armar objetos usando distintas arquitecturas y algunas
%       que se puedan citar que alteran la función de pérdida, como
%       FactorVAE y otros. Citar un paper que pueda entender,
%       quizá FactorVAE no sea el mejor para esto.
que tratan de separar el problema de la reconstrucción de un objeto entre
las partes que lo componen.

% TODO: acá mismo relatar cómo es un GAN, por qué es generativo, por qué
%       es difícil de entrenar (inestabilidad), qué viene a resolver
%       \cite{pmlr-v70-arjovsky17a}
%       En el libro de Goodfellow et al. "Deep Learning", en la página
%       699 empieza el apartado de las redes GAN.

Otra aproximación a la reconstrucción de objetos es la estrategia implementada
por las redes GAN, ``Generative adversarial networks''. Este tipo de
arquitectura está basada en un juego minimax % TODO: agregar referencia a los juegos minimax de algún lado que lo describa
en el cual dos redes, una generadora de objetos encargada de generar objetos
lo más similar posible a los objetos del conjunto de datos de entrenamiento
y una red discriminadora encargada de distinguir entre objetos generados y
reales, compiten entre sí. La red generadora debe generar objetos que la red
no pueda distinguir de un objeto real (ergo, que clasifique al objeto como
real cuando es generado; entonces se dice que la generadora \textit{engaña} a
la red discriminadora) y la red discriminadora debe poder realizar la distinción.
\cite{goodfellow2016deep}. En sucesivas iteraciones de entrenamiento ambas redes 
refinan su capacidad de generar y discriminar, respectivamente. Lo esperado en
la convergencia de este juego es que los objetos generados sean prácticamente
indistinguibles de los reales, obteniendo una red generadora de objetos, y que
el discriminador no pueda realizar la distinción, es decir, que la probabilidad
de que un objeto sea real o generado sea de 0,5.
No obstante, el entrenamiento de una red GAN es difícil, dado que es una red
muy permeable a caer en mínimos locales, no ajustando lo suficiente como para
obtener soluciones aceptables u óptimas debido a la naturaleza de los juegos
minimax, en el cual se pueden llegar a mínimos locales para las dos redes poco
óptimos, en donde una red puede adoptar una estrategia que le permita ganar el
juego a costa de un desempeño mediocre. Este tipo de resultados se lo denomina
``mode collapse'', en donde una de las redes aprende muy rápidamente a ganar
siempre el juego antes de que se haya realizado algún aprendizaje por parte de
la otra red.  Una propuesta para impedir esta inestabilidad en el entrenamiento
de las redes GAN lo proponen \cite{pmlr-v70-arjovsky17a} en donde incluyen una componente
de pérdida para el discriminador, un ``crítico'', el cual se basa en la distancia
Wasserstein entre las distribuciones de probabilidad a priori y de los datos,
componente que se adiciona a la función de pérdida del generador.Si el
discriminador aprende más rápido y no proporciona información suficiente para
actualizar los gradientes del generador, el crítico sí permite hacerlo y mejora
sustancialmente el desempeño y la protección contra el ``mode collapse''.

\subsection{Redes neuronales sobre objetos tridimensionales}
Las redes convolucionales que tratan con datos tridimensionales se
han utilizado para múltiples objetivos. \cite{savva2016large, savva2017large}
comenta diversos trabajos de clasificación para buscadores de objetos
tridimensionales sobre datos \textit{benchmark}, en estos trabajos,
los datos de ShapeNet Core 55 \cite{chang2015shapenet}. Para este
tipo de trabajos de clasificación se usaron redes neuronales que
trabajaron sobre todo tipo de datos, mostrando la versatilidad de
las redes neuronales sobre distintos tipos de representaciones de
datos.
Las distintas representaciones de datos conllevan distinto tipo
de arquitecturas para poder trabajarlas. Las convoluciones
anteriormente explicadas funcionan muy bien para datos con formato
matricial, como ser objetos bidimiensionales como las imágenes,
representadas por píxeles en capas de distintos canales,
que comúnmente representan colores. El tipo de dato que mejor se
adapta al uso de convoluciones en datos tridimensionales es el
vóxel, el cual representa un homólogo al píxel bidimensional.
También sirven de manera adecuada para el uso de datos de objetos
representados como múltiples vistas bidimensionales \cite{arsalan2017synthesizing}.
Otros tipos de datos que no se prestan directamente al uso de redes convolucionales
son las mallas de polígonos \cite{Gao2019}, los puntos \cite{Qi2017, qi2017pointnet}
y las representaciones implícitas \cite{Park2019}.

% TODO: Leer y agregar cómo se comportan los datos de puntos en PointNet

\subsection{Arquitecturas de Deep Learning en 3D}

En \cite{Muralikrishnan2019}, se utiliza la arquitectura VAE y la
representación de objetos tridimensionales en forma de voxeles,
puntos y multiples vistas 2D para construir un mismo espacio de códigos
latentes. Esto permite que, en \textit{testing}, se puede proveer a la
arquitectura datos de cualquiera de los tres tipos mencionados,
construyendo luego las otras dos representaciones del objeto. Los datos
mencionados se obtienen a partir de mallas poligonales transformadas en
representaciones de los tres tipos mencionados.
\cite{Muralikrishnan2019} reporta que la arquitectura propuesta consigue
aislar la representación del objeto respecto del objeto mismo, consiguiendo
un código latente con mayor información del objeto en sí, independientemente
del tipo de dato utilizado como representación.
La arquitectura que proponen consiste en un VAE por cada una de las tres
representaciones tratadas. Cada uno de los decoders reconstruye sólo una
de las tres representaciones.
Por otro lado, la función de pérdida es central para poder lograr la
traducción entre representaciones distintas de un mismo objeto. En primer
lugar, la función de pérdida favorece que el código latente formado por un
mismo objeto sea el mismo independientemente de la representación de entrada
de dicho objeto. Asimismo, no incluyen la divergencia de Kullback-Leibler
en la función objetivo, debido a que limitaría la separación de grupos
(produciéndose un solapamiento de los mismos en el espacio de los códigos
latentes). Esto podría provocar discontinuidades en dicho espacio,
dificultando la interpolación entre códigos para la generación de nuevos
objetos no observados anteriormente por la red. Por otra parte, para el
cómputo de la pérdida de reconstrucción se utiliza una función distinta
para cada una de las tres distintas representaciones, adecuada a cada una
de ellas. Por último, a la salida de los decoders se utiliza una capa
completamente conectada para realizar tareas de discriminación de las
categorías de objetos utilizadas para entrenar, resultando en la última
componente utilizada para la función de pérdida.

%%%%%%%%%%%%%%%%%%%%%%%%%%%%%%%%%%%%%%%%%%%%%%%%%%%%%%%%%%%%%%%%%%       
%%%%% Apartado nuevo debatiendo las ideas de SDF de Park2019 %%%%%
%%%%%%%%%%%%%%%%%%%%%%%%%%%%%%%%%%%%%%%%%%%%%%%%%%%%%%%%%%%%%%%%%%       
\cite{Park2019} propone una red que representa los objetos tridimensionales
en un campo volumétrico cuyo valor en cualquier punto representa la 
distancia de ese punto a la superficie del objeto, y su signo representa si
el punto se encuenta en el interior o en el exterior del objeto. De esta
forma, la superficie se encuentra en el valor cero devuelto por la función
aprendida por la red, llamada \textit{SDF} (\textit{SignedDistance
Function} por sus siglas en inglés).

Los datos usados para entrenar el modelo consisten en un muestreo de puntos
cercanos a la superficie de objetos representados como mallas de triangulos.
Luego, a cada punto se le aplica la función SDF, obteniendo un valor para
cada punto, siendo la combinación de los puntos y su valor de SDF los datos
de entrenamiento.

La arquitectura entrenada consiste en un \textit{autodecoder}: solo el
código latente y el decoder de un tradicional autoencoder es usado. Para
cada uno de los elementos del dataset de entrenamiento, se inicializa un
código latente aleatorio que luego es modificado por el algoritmo de
\textit{backpropagation}. La forma de operar es similar a un
\textit{autoencoder}, el codigo latente representa una codificación de SDF,
que luego es utilizado para reconstruir cada objeto usando la función SDF
inferida por la red. Como resultados comparables con otro tipo de redes,
la arquitectura propuesta logra captar detalles finos de los objetos y
puede reconstruirlos con una alta resolucion. No obstante, en objetos
muy detallados puede fallar en reconstruirlos. Otra de los logros de la
red es poder disminuir el tamaño en memoria de los objetos y del modelo
debido a la elección de representación de los mismos elegida. Esto hace
posible un entrenamiento más rápido y menores necesidades computacionales
que otras representaciones para una misma representación. No obstante,
\cite{Park2019} reporta que la forma habitual de entrenamiento de un
VAE no les ha dado resultado para este tipo de problemas, por lo que
podría suponerse que esta red tiene problemas para generar nuevos objetos % Esta parte es suposición, no sé si conviene dejarla.
creando nuevos códigos latentes para alimentar el \textit{autodecoder}.
La forma de entrenamiento tampoco permite conocer la distribución a
posteriori del espacio de códigos latentes del dataset de entrenamiento.
Otro de los problemas al intentar usar esta red es debido a que la red
entrenada es específica de los datos usados, ya que no hay
\textit{encoder} que permita generar nuevos codigos latentes a partir
de nuevos datos. Esto es, la red aprende solo una representación
comprimida de los datos de entrenamiento.

%%%%%%%%%%%%%%%%%%%%%%%%%%%%%%%%%%%
%%%%% Apartado viejo Park2019 %%%%%
%%%%%%%%%%%%%%%%%%%%%%%%%%%%%%%%%%%
% También existen otras formas de representaciones con performances
% destacables, como eluso de funciones de distancia con signo \cite{Park2019}
% (Signed Distance Functions, SDF, por sus siglas en inglés) en el que
% se muestrean puntos cerca de la superficie de un objeto teniendo en
% cuenta si el punto se encuentra dentro o fuera de la figura
% (representado por el signo) y la distancia más corta del punto hacia
% la superficie del objeto a entrenar. Esto permite que se calcule una
% SDF por cada una de las figuras de entrenamiento, teniendo luego un
% auto encoder (o auto decoder, un tipo de arquitectura que no usa
% encoders para entrenar, como en el caso de \cite{Park2019}) que
% realizar una inferencia de las SDF de cada figura y encontrar
% abstracciones entre las distintas SDF de cada figura de los datos de
% entrenamiento.

%%%%%%%%%%%%%%%%%%%%%%%%%%%%%%%%%%%%%%%%%%%%%%%%%%%%%%%%%%%%%%%%%%%%%%%
%%%%% Apartado nuevo debatiendo las ideas de TBN de Olszewski2019 %%%%%
%%%%%%%%%%%%%%%%%%%%%%%%%%%%%%%%%%%%%%%%%%%%%%%%%%%%%%%%%%%%%%%%%%%%%%%
TBN \cite{Olszewski2019} es una red que infiere objetos tridimensionales
a partir de vistas bidimensionales y genera nuevas vistas del mismo,
realizando transformaciones sobre el espacio tridimensional aprendidas
por la red al incluir las transformaciones junto a los datos de
entrenamiento.

Los datos de entrenamiento consisten en imágenes a transformar en
tridimensionales así como también las transformaciones a aplicar a
los objetos. De esta manera, se espera que la red aprenda tanto a
transformar un objeto bidimensional a tridimensional así como también a
aplicar las transformaciones sobre los mismos para generar nuevas vistas
bidimensionales (NVS, o \textit{Novel View Synthesis}, por sus siglas en
inglés).

La arquitectura consiste en un \textit{autoencoder} modificado, el cual
conserva las partes \textit{encoder-decoder} e incluye un
\textit{bottleneck}, o cuello de botella, entre ambos, en donde ocurren
las manipulaciones espaciales de los objetos tridimensionales. Estas
transformaciones pueden ser variadas, pudiendo interactuar con un objeto
de diversas formas. Otra de las diferencias con un \textit{autoencoder}
tradicional reside en que el \textit{encoder} realiza la transformación
de un espacio bidimensional a uno tridimensional. Tanto el \textit{encoder}
como el \textit{decoder} son redes convolucionales pero contienen un paso
intermedio de \textit{reshaping} para el cambio de dimensiones. Por ende,
el \textit{bottleneck} contiene un objeto tridimensional, no un código
latente. Además, el usuario provee la función de transformación para este
objeto en el \textit{bottleneck}. 
La funcion de pérdida utilizada para entrenar la red tiene varias
componentes:
\begin{itemize}
    \item Pérdida de reconstucción \(L_{1}\).
    \item Pérdida perceptual, definida como la pérdida en el espacio de
        características de la red VGG-19 \cite{simonyan2014vgg19}
    \item Pérdida de similitud %TODO: Revisar las citas que tiene respecto
                               % de este tipo de pérdida, para poder anotar
                               % la matemática de la misma.
    \item Pérdida adversarial, utilizando el discriminador de una red GAN.
    \item Pérdida de segmentación: para cada una de las imágenes, tanto el
        \textit{input} como el \textit{output} tienen una máscara binaria
        (ceros y unos) que identifican los píxeles donde se encuentra el
        objeto con 1 y el resto con 0. De esta forma, se puede medir el
        resultado de la segmentación lograda por el decoder.
\end{itemize}

\begin{figure}[h]
\includegraphics[height=4cm]{images/Olszewski2019_arq.png}
\centering
\caption{Arquitectura de TBN \cite{Wu2016} para NVS usando un \textit{bottleneck} tridimensional}
\label{TBN_arq}
\end{figure}

%%%%%%%%%%%%%%%%%%%%%%%%%%%%%%%%%%%%%%%%%%%%%%%%%%%%%%%%%%%%%%%%%%%%%%%
%%%%% Apartado viejo debatiendo las ideas de TBN de Olszewski2019 %%%%%
%%%%%%%%%%%%%%%%%%%%%%%%%%%%%%%%%%%%%%%%%%%%%%%%%%%%%%%%%%%%%%%%%%%%%%%
% Hay nuevas arquitecturas \cite{Olszewski2019} que
% también han logrado generar vistas nuevas de objetos tridimensionales
% (Novel View Synthesis, o NVS por sus siglas en inglés). En el caso de
% \cite{Olszewski2019}, se entrena usando varios objetos con múltiples
% vistas (imágenes en dos dimensiones) de un mismo objeto tridimensional
% en una arquitectura con encoder, un "bottleneck" (o "cuello de botella")
% y un decoder para recrear la imagen con las transformaciones aplicadas
% en el bottleneck. Lo interesante del método empleado consiste en que la
% NVS no es algo inferido por la red en el entrenamiento, sino que es
% aplicado de una forma no aprendida por la red en el bottleneck, lo que
% permite realizar otras transformaciones además de nuevas vistas de un
% objeto (como, por ejemplo, distintas poses de un ser humano) sin variar
% la forma de entrenar la red. Asimismo, las interpolaciones son
% diferenciables y permiten el correcto funcionamiento del algoritmo de
% backpropagation para poder actualizar los parámetros del encoder.

%%%%%%%%%%%%%%%%%%%%%%%%%%%%%%%%%
%%%%% Apartado nuevo Li2019 %%%%%
%%%%%%%%%%%%%%%%%%%%%%%%%%%%%%%%%
\textit{Part Aware Generative Network}, PAGENet \cite{Li2019}, es una
arquitectura que consiste en varias redes que actuan sobre las partes
de una serie de objetos tridimensionales voxelizados y segmentados
semánticamente en sus partes, con otra red encargada del ensamblado
de las partes para obtener la reconstrucción de los datos de entrada.
Con este tipo de arquitectura se pretende solucionar un problema
recurrente en la reconstrucción de figuras en zonas de alta varianza
en los datos, correspondiente en general a detalles más finos, sin
incrementar la resolución.
La arquitectura de PAGENet consiste en varias partes. En primer lugar
están los generadores por parte, que consisten en redes VAE-GAN, cuya
misión es reconstruir las partes semánticas del objeto.
Para la reconstrucción de las partes se usa una función de pérdida que
incluye las siguientes componentes:
* Pérdida de reconstrucción.
* Divergencia Kullback-Leibler.
* Pérdida de reflexión: para las partes simétricas, se computa esta
pérdida si la reconstrucción no lo es.
* Pérdida adversarial: Correspondiente al discriminador de la red GAN.
El VAE actúa como generador.

Usando los objetos reconstruidos de estas redes se encuentra otra red
encargada del ensamblado de las partes generadas previamente,
trabajando directamente en el ensamblado de los objetos.
Como las redes individuales en ningun momento tratan el tamaño
adecuado de las partes para lograr la correcta
reconstrucción del objeto, de ello se tiene que encargar la
arquitectura de ensamblado, encargándose de todas las
transformaciones necesarias a cada una de las partes. Esta
red tiene cinco capas convolucionales con tamaños de
kernel de \(4x4x4\) junto con las habituales capas de normalización
por lotes y capas ReLU. Por lo tanto, el output no es directamente
un objeto, sino una serie de transformaciones a aplicar a
las partes recibidas de las redes generadoras para finalmente obtener
el objeto reconstruido. Asimismo, para dar variedad a los objetos
producidos, se puede dejar constante en 1 el valor de las
transformaciones de una de las partes, llamada \textit{ancla}, para
si transformar las otras y que se amolden al ancla. Esto hace
que en cada momento haya tantas posibles reconstrucciones de un mismo
objeto como partes haya, ya que cualquiera de ellas puede elegirse
como ancla.

%%%%%%%%%%%%%%%%%%%%%%%%%%%%%%%%%
%%%%% Apartado viejo Li2019 %%%%%
%%%%%%%%%%%%%%%%%%%%%%%%%%%%%%%%%
% \cite{Li2019} logra una mejor reconstrucción de figuras en tres dimensiones
% si se reconstruyen las partes por separado (teniendo datos de entrenamiento
% con partes claramente segmentadas, con el fin de poder dividirlas y usar
% distintos modelos para aprenderlas) y luego se ensamblan de acuerdo a la
% forma en que se corresponden las distintas partes, pudiendo tener leves
% variaciones respecto a las piezas originales pero correspondiéndose las
% partes entre sí a la hora del ensamble. Asimismo, eligiendo una de las
% piezas como ancla para el ensamble, pueden lograr generar distintas
% figuras dentro del espacio de figuras estudiado. La arquitectura que usan
% corresponde a un VAE-GAN: un variational autoencoder cuya reconstrucción
% de salida es la entrada de una red GAN. Esto genera un código latente
% formado por los códigos de las partes individuales, lo que alimenta una
% última red que es la encargada del ensamble de las partes de la figura.

\begin{figure}[h]
\includegraphics[height=4cm]{images/Li2019_arq.png}
\centering
\caption{Arquitectura de PAGENet \cite{Li2019} para la generación de objetos tridimensionales ensamblando sus partes}
\label{PAGENet_arq}
\end{figure}

Existen otras formas de lograr representaciones tridimensionales sin 
usar datos en forma de vóxeles. En \cite{Gao2019} las redes aprenden a
deformar una malla de triángulos homeomórfica a una caja patrón para
reconstruir las partes de una figura, capturando mayor grado de detalles
y teniendo una continuidad superior en la superficie de los objetos que
usando vóxeles o nubes. Luego, una segunda red toma las partes reconstruidas
y las ensambla de una forma coherente, reconstruyendo el objeto final.
La arquitectura propuesta, llamada SDM-NET
(\textit{Structured Deformable Meshes}, por sus siglas en inglés),
consta de dos VAE: la primera red, que llaman
\textit{PartVAE} se entrena sobre un dataset con figuras etiquetadas
por partes, por lo que aprende a reconstruir las partes por separado.
La segunda, que llaman \textit{SP-VAE} (por ``Structured Parts VAE'')
aprende la estructura global del objeto, asegurando luego la coherencia
de las partes ensambladas.
La principal limitación de \textit{SDM-Net} se da tanto en el tipo de datos.
Al estar los objetos constituidos por mallas de triángulos, no se pueden
aplicar convoluciones de la misma manera que al usar vóxeles para preservar
los objetos. Asimismo, la representación de los objetos de entrenamiento
no es con mallas directamente sino con \textit{mallas deformables}. Es decir,
primero se debe establecer una caja patrón para las partes de los objetos con
una cierta malla característica, sobre la cual se debe luego aplicar las
deformaciones para obtener los objetos originales. Este tipo de datos
preprocesados es el que alimentan la primera red, \textit{PartVAE}. Luego, para
dotar de coherencia al código latente generado por \textit{PartVAE}, agregan
más información al código latente sobre el objeto completo, ya que la información
de las partes reconstruidas no garantiza de ninguna forma que haya una coherencia
en el objeto final reconstruido por \textit{SP-VAE}. La información agregada al
código latente incluye información de simetría entre partes, relaciones de soporte
y de tamaño relativo entre las partes, entre otras. Esta información adicional
no surge de la primera red, sino que surge del preprocesamiento de los códigos
latentes generados por \textit{PartVAE} antes de entrar en la red \textit{SP-VAE}.
Por todo esto se aumenta la complejidad del manejo de datos respecto a utilizar
vóxeles para la representación de objetos.

\cite{Yin2019} propone una arquitectura llamada LOGAN, por sus siglas en inglés de
\textit{Latent Overcomplete Generative Adversarial Network} con el propósito de
lograr traducciones de figuras entre dos dominios distintos pero con características
similares, como pueden ser sillas y mesas, de una forma no supervisada, es decir,
no hay ninguna segmentación previa, ninguna correspondencia entre puntos realizada
entre figuras de los dos dominios ni ningún tipo de emparejamiento de los dos
dominios previo al entrenamiento, sino a ser encontrado por la red luego de
terminado el entrenamiento.
La arquitectura está compuesta por dos partes, un autoencoder y dos traductores que
van en direcciones opuestas: uno traduce del dominio \textit{x} al \textit{y} y el
otro, en el sentido opuesto. Estos traductores tienen forma de red GAN. Tanto los
autoencoders como los traductores son entrenados de forma independiente entre sí
pero siguiendo un orden: primero se entrena el autoencoder y luego, los traductores.
La particularidad en el entrenamiento del autoencoder consiste en que los vectores
latentes se forman a partir de las capas PointNet++ \cite{Qi2017} del encoder, de
forma tal que forman un vector con la concatenación de las partes (tipo
\textit{overcomplete}) y otros vectores con las partes individuales rellenos de
ceros para completar el tamaño de vector necesario para alimentar el decoder. A su
vez, se alimenta el decoder con todos estos vectores, de forma tal que la red puede
aprender distintas características de forma independiente, lo que lleva a un
\textit{disentangling} implícito en las partes que conforman el vector
\textit{overcomplete}. Como función de pérdida se usa EMD (Earth Mover's Distance,
por sus siglas en inglés). Luego se entrenan por separado las redes traductoras,
cuya función de pérdida es algo más compleja para permitir la correcta traducción
entre los dominios. La primera componente es la función de pérdida adversarial de
la WGAN \cite{pmlr-v70-arjovsky17a} a la que se suma una componente de preservación
de características, que mide el error de traducción entre una figura de un dominio
\textit{x} a su mismo dominio, teniendo que replicar la misma figura. Esto permite
que la red aprenda a preservar las caracteristicas importantes de la figura y a que
la traducción al dominio \textit{y} preserve los aspectos más dominantes de la
figura, haciendo sólo algunos cambios en la traducción. Por último, lo que detallan
como una componente menor sólo para prevenir que un grupo de figuras del dominio
\textit{y} abarquen todas las traducciones desde el dominio \textit{x}, agregan una
última componente llamada \textit{de ciclo}, la cual comprende la pérdida de volver
a traducir una figura desde \textit{y} hacia \textit{x} que previamente había sido
traducida en el sentido contrario, teniendo que volverse a reconstruir la figura
original, forzando a que haya más relaciones unívocas entre dominios.  

\begin{figure}[h]
\includegraphics[height=4cm]{images/Yin2019_arq_autoencoder.png}
\centering
\caption{Arquitectura de autoencoder en LOGAN \cite{Yin2019} para la generación de objetos tridimensionales a partir de códigos latentes \textit{overcomplete}}
\label{LOGAN_arq_ae}
\end{figure}

% TODO: Revisar aca si ponemos el caso de MNIST que probé, sería lo más sensato
%       En el paper de SDF hacen eso, ponen MNIST para mostrar por que usan un auto-decoder.

\clearpage

\section{Método}%
\label{sec:metodo}

\subsection{Arquitectura}%
\label{sub:arquitectura}
La arquitectura que se utilizará como base es el Variational Autoencoder, para
poder entrenar y generar luego nuevos objetos. Una variante que se realizará
de esta arquitectura base está inspirada por la red ResNET \cite{He_2016}, la
cual mejora el rendimiento de las redes muy profundas ajustando residuos de
funciones que ajustan a los datos correctamente en lugar de solo apilar capas
intermedias.

La forma de entrenamiento

\subsection{Experimental setup}%
\label{sub:experimental_setup}

Entorno de entrenamiento: Google Colaboratory, Notebooks de Jupyter (Lenguaje de Programación Python).
\begin{itemize}
    \item Device: docker Mobo: Google model: Google Compute Engine serial: Board-GoogleCloud-D1F1803758000E0AA854AF5459F55D7E
    \item CPU: Intel(R) Xeon(R) CPU, 2.00 GHz, 2 núcleos (4 núcleos lógicos).
    \item OS: Ubuntu 18.04.3 LTS
    \item RAM: 12.72 GB
    \item GPU: NVIDIA Corporation TU104GL [Tesla T4] (rev a1)
\end{itemize}

% inxi output
% System:    Host: 8f66488bf791 Kernel: 4.19.104+ x86_64 bits: 64
%            Console: tty 0 Distro: Ubuntu 18.04.3 LTS
% Machine:   Device: docker Mobo: Google model: Google Compute Engine serial: Board-GoogleCloud-D1F1803758000E0AA854AF5459F55D7E
%            BIOS: Google v: Google date: 01/01/2011
% Battery    Using dmidecode: unknown error occurred
% CPU:       Single core Intel Xeon (-MT-) arch: Skylake rev.3 cache: 39424 KB
%            flags: (lm nx sse sse2 sse3 sse4_1 sse4_2 ssse3) bmips: 4000
%            clock speeds: max: 2000 MHz 1: 2000 MHz 2: 2000 MHz
% Graphics:  Card: NVIDIA Device 1eb8 bus-ID: 00:04.0
%            Display Server: X.org 1.20.5 driver: nvidia
%            tty size: 0x0 Advanced Data: N/A for root out of X
% Network:   Card: Red Hat Virtio network device
%            driver: virtio-pci port: c000 bus-ID: 00:05.0
%            IF: N/A state: N/A speed: N/A duplex: N/A mac: N/A
% Drives:    HDD Total Size: 85.9GB (40.7% used)
%            ID-1: /dev/sda model: PersistentDisk size: 85.9GB
% Partition: ID-1: / size: 69G used: 32G (49%) fs: overlay dev: N/A
% RAID:      No RAID devices: /proc/mdstat, md_mod kernel module present
% Sensors:   None detected - is lm-sensors installed and configured?
% Info:      Processes: 9 Uptime: 16 min Memory: 743.2/13021.1MB
%            Init: N/A Gcc sys: 7.5.0 Client: Shell (python3) inxi: 2.3.56 

% lspci output
% 00:00.0 Host bridge: Intel Corporation 440FX - 82441FX PMC [Natoma] (rev 02)
% 00:01.0 ISA bridge: Intel Corporation 82371AB/EB/MB PIIX4 ISA (rev 03)
% 00:01.3 Bridge: Intel Corporation 82371AB/EB/MB PIIX4 ACPI (rev 03)
% 00:03.0 Non-VGA unclassified device: Red Hat, Inc. Virtio SCSI
% 00:04.0 3D controller: NVIDIA Corporation TU104GL [Tesla T4] (rev a1)
% 00:05.0 Ethernet controller: Red Hat, Inc. Virtio network device
% 00:06.0 Unclassified device [00ff]: Red Hat, Inc. Virtio RNG

% Architecture:        x86_64
% CPU op-mode(s):      32-bit, 64-bit
% Byte Order:          Little Endian
% CPU(s):              2
% On-line CPU(s) list: 0,1
% Thread(s) per core:  2
% Core(s) per socket:  1
% Socket(s):           1
% NUMA node(s):        1
% Vendor ID:           GenuineIntel
% CPU family:          6
% Model:               85
% Model name:          Intel(R) Xeon(R) CPU @ 2.00GHz
% Stepping:            3
% CPU MHz:             2000.170
% BogoMIPS:            4000.34
% Hypervisor vendor:   KVM
% Virtualization type: full
% L1d cache:           32K
% L1i cache:           32K
% L2 cache:            1024K
% L3 cache:            39424K
% NUMA node0 CPU(s):   0,1
% Flags:               fpu vme de pse tsc msr pae mce cx8 apic sep mtrr pge mca cmov pat pse36 clflush mmx fxsr sse sse2 ss ht syscall nx pdpe1gb rdtscp lm constant_tsc rep_good nopl xtopology nonstop_tsc cpuid tsc_known_freq pni pclmulqdq ssse3 fma cx16 pcid sse4_1 sse4_2 x2apic movbe popcnt aes xsave avx f16c rdrand hypervisor lahf_lm abm 3dnowprefetch invpcid_single ssbd ibrs ibpb stibp fsgsbase tsc_adjust bmi1 hle avx2 smep bmi2 erms invpcid rtm mpx avx512f avx512dq rdseed adx smap clflushopt clwb avx512cd avx512bw avx512vl xsaveopt xsavec xgetbv1 xsaves arat md_clear arch_capabilities

Como herramienta principal para la construcción de las redes en el entorno
generado por Google Colab descrito anteriormente se utilizó la librería
Pytorch \cite{PytorchNEURIPS2019_9015}.

\subsection{Métodos competidores}%
\label{sub:metodos_competidores}
Como métodos que compiten con el modelo a presentar en este
trabajo, se encuentran la red PAGENet \cite{Li2019} que ajusta una red
VAE-GAN por cada una de las partes de los objetos de entrenamiento,
usando luego una red ensambladora al final para la reconstruccion de
los mismos. Otra de las redes que trata un dataset segmentado por partes
basado en ShapeNet es \cite{G2L18}, basado en la arquitectura GAN mejorada
por \cite{GulrajaniAADC17} en la cual, además de un discriminador global,
utilizan también un discriminador por cada parte de los objetos. Asimismo,
cuenta con un autoencoder que denominan \textit{Part Refiner}, con el que
aumentan la resolución de las partes, en vóxeles de $32^3$ a $64^3$ y
disminuyen los artefactos y partes desconectadas de las partes generadas,
para un mejor ensamble en la reconstrucción del objeto final.

\subsection{Función objetivo}%
\label{sub:funcion_objetivo}

% TODO: Adaptar a lo que quede como objetivo para la función final, sobre
%       todo si se usan losses distintas a las comunes
Las funciones objetivos de las distintas redes que presentaremos a
continuación tienen distintas componentes que cumplen diversos fines para
el entrenamiento de las redes. Estas componentes a utilizar son las
siguientes:
\begin{itemize}
    \item Divergencia de Kullback-Leibler \cite{Joyce2011KullbackLeiblerD}: consiste
        en la divergencia entre dos distribuciones. Se utilizará como una componente
        de la función de pérdida para que a red tienda a ajustar la distribución de
        cada componente de los vectores latentes a una distribución normal estándar,
        tendiendo los vectores latentes a una distribución estándar normal
        multivariada. De esta forma, se permite una continuidad en el espacio de los
        objetos representados por los códigos latentes que con un autoencoder
        tradicional es imposible lograrlo, permitiendo luego la generación de nuevos
        objetos mediante el muestreo sobre el espacio generado.
    \item \textit{Binary cross-entropy}: Entre la entrada y la salida. Sirve para
        medir la calidad de la reconstrucción del objeto.
    \item Pérdida L1 suave: la pérdida L1, conocida también error absoluto medio, es
        igual al valor absoluto de la diferencia entre el valor predicho y el valor
        de entrada. En el caso de la versión suave, sigue la siguiente función por
        partes:
        \[
            z_i =
            \begin{cases}
                0.5(x_i - y_i)^2, &\quad\text{if } \lvert x_i - y_i \lvert < 1  \\
                
                \lvert x_i - y_i \lvert - 0.5, &\quad\text{de otra manera.} \\
            \end{cases}
        \]
\end{itemize}
Las distintas componentes de la función de pérdida también pueden ser pesadas por
pesos $w_i$, los cuales serán variables para poder medir los distintos efectos de
cada uno de los componentes de esta función.
De esta forma la función de pérdida toma la forma:
\[
    \mathcal{L}_{total} = w_{KL} * \mathcal{D}_{KL} + w_{BCE} * \mathcal{L}_{BCE} + w_{L1} * \mathcal{L}_{L1}
\]

Éstas componentes, así como la pérdida total,se reportarán en las evaluaciones
que se hagan de los modelos.
También se usarán los siguientes hiperparámetros:

\begin{itemize}
    \item Tasa de aprendizaje = 0.001
    \item Épocas: variable
    \item Dimensiones de los códigos latentes: variable.
    \item pesos de las componentes de la función de pérdida: variables.
\end{itemize}

Las capas que se utilizarán son capas convolucionales como fue descrito en
"\textit{\nameref{sec:estado_del_arte}}", donde las capas convolucionales pueden
entenderse en conjunto con las capas de normalización por lotes y las capas ReLU.
También pueden utilizarse las capas tradicionales completamenete conectadas,
como en las capas que generan los códigos latentes, normalmente con activaciones
acotadas, como la función sigmoidea.
Por último, se utilizará la técnica de \textit{dropout}, en la cual se suelen
desactivar algunas neuronas en ciertas iteraciones para aumentar la capacidad de
generalización de la red.

\section{Evaluación}%
\label{sec:evaluacion}
Como primeros experimentos se probó la arquitectura VAE sobre datos estándares
del área del procesamiento de imágenes usando redes profundas, en este
primer caso, MNIST \cite{LeCun1998GradientbasedLA}. Este conjunto de datos se
basa en el conjunto de datos NIST, que consta de unas 60.000 imágenes de
números del 0 al 9 dibujados a mano, los cuales luego fueron modificados para
evitar el preprocesamiento de estas imágenes, dando lugar a MNIST. \\
Para los diagramas que se representarán de ahora en adelante, se tomará la
siguiente codificación para la representación de las capas de la red:
\begin{itemize}
    \item C: Convolución.
    \item AP: \textit{Average pooling}, ``Pooling'' promedio; toma el valor
        promedio dentro del kernel definido.
    \item R: ReLU, \textit{Rectified Linear Unit}.
    \item DO: \textit{dropout}, desconexión de algunas de las neuronas para
        mejorar la generalización de la red.
    \item BN: \textit{Batch normalization}, normalización por lotes.
    \item MP: \textit{Max pooling}, ``Pooling'' máximo, toma el valor máximo
        dentro del kernel definido.
    \item CT: Capa convolucional traspuesta, utilizada como operación inversa
        a la convolución, utilizada en el decoder.
    \item LC: Código latente, código formado por un encoder.
    \item O: salida de la red.
    \item FC: \textit{fully connected}, capa de neuronas completamente
        conectadas.
\end{itemize}
Asimismo, para cada una de las capas se utilizará un color para identificarlas,
como puede verse en la figura \ref{fig:ref_nn_diag}.

\shorthandoff{<}
\shorthandoff{>}
\begin{figure}[htpb]
\begin{center}
    \begin{tikzpicture}[scale=1,node distance=3cm,outer sep=10pt]
    \tikzstyle{conv}=[minimum width=0.2cm,minimum height=1cm,draw=white,fill=blue!20,inner sep=0]
    \tikzstyle{convt}=[minimum width=0.2cm,minimum height=1cm,draw=white,fill=blue!75,inner sep=0]
    \tikzstyle{bn}=[minimum width=0.2cm,minimum height=1cm,draw=white,fill=yellow!75,inner sep=0]
    \tikzstyle{relu}=[minimum width=0.2cm,minimum height=1cm,draw=white,fill=teal!75,inner sep=0]
    \tikzstyle{avg}=[minimum width=0.2cm,minimum height=1cm,draw=white,fill=green!75,inner sep=0]
    \tikzstyle{fc}=[minimum width=0.2cm,minimum height=1cm,draw=white,fill=black!75,inner sep=0]
    \tikzstyle{do}=[minimum width=0.2cm,minimum height=1cm,draw=white,fill=red!75,inner sep=0]
    \tikzstyle{latent code}=[minimum width=0.5cm,minimum height=1.5cm,thick,draw=black!75,fill=white]
    \tikzstyle{output}=[minimum width=0.5cm,minimum height=1.5cm,thick,draw=orange!75,fill=orange!30]
    \node [conv]    (conv1)                       {};
    \node [convt]   (convt1)    [right of=conv1]  {};
    \node [bn]      (bn1)       [right of=convt1] {};
    \node [relu]    (relu1)     [right of=bn1]    {};
    \node [fc]      (fc1)       [right of=relu1]  {};
    \node [avg]     (avg1)      [below of=conv1]  {};
    \node [do]      (do1)       [right of=avg1]   {};
    \node [latent code] (lc)    [right of=do1]    {};
    \node [output]  (out1)      [right of=lc]     {};
    \node at (conv1.south)  {C};
    \node at (convt1.south) {CT};
    \node at (bn1.south)    {BN};
    \node at (relu1.south)  {R};
    \node at (fc1.south)    {FC};
    \node at (avg1.south)   {A};
    \node at (do1.south)    {DO};
    \node at (lc.south)     {LC};
    \node at (out1.south)   {O};
\end{tikzpicture}
\end{center}
\caption{Referencia diagrama de redes neuronales}%
\label{fig:ref_nn_diag}
\end{figure}
\shorthandon{<}
\shorthandon{>}

En la figura \ref{fig:mnist_vae} mostramos cómo es la arquitectura que
utilizaremos para hacer el primer modelo para generar figuras del
conjunto de datos MNIST.

\shorthandoff{<}
\shorthandoff{>}
\begin{figure}[htpb]
\begin{center}
\begin{tikzpicture}[scale=1,auto,node distance=0.2cm,font=\tiny]
    \tikzstyle{invisible}=[minimum width=0mm,inner sep=0mm,outer sep=0mm]
    \tikzstyle{conv}=[minimum width=0.2cm,minimum height=1cm,draw=white,fill=blue!20,inner sep=0]
    \tikzstyle{convt}=[minimum width=0.2cm,minimum height=1cm,draw=white,fill=blue!75,inner sep=0]
    \tikzstyle{bn}=[minimum width=0.2cm,minimum height=1cm,draw=white,fill=yellow!75,inner sep=0]
    \tikzstyle{relu}=[minimum width=0.2cm,minimum height=1cm,draw=white,fill=teal!75,inner sep=0]
    \tikzstyle{fc}=[minimum width=0.2cm,minimum height=1cm,draw=white,fill=black!75,inner sep=0]
    \tikzstyle{avg}=[minimum width=0.2cm,minimum height=1cm,draw=white,fill=green!75,inner sep=0]
    \tikzstyle{do}=[minimum width=0.2cm,minimum height=1cm,draw=white,fill=red!75,inner sep=0]
    \tikzstyle{latent code}=[minimum width=0.5cm,minimum height=1.5cm,thick,draw=black!75,fill=white]
    \tikzstyle{output}=[minimum width=0.5cm,minimum height=1.5cm,thick,draw=orange!75,fill=orange!30]
    \tikzstyle{pre}=[<-,>=stealth',semithick]
    \tikzstyle{post}=[->,>=stealth',semithick]
    \node   [invisible] (input)                   {};
    \node   [conv] (conv1) [right=1.5cm of input] {}
        edge [pre] node [below] {$b \text{x} 1 \text{x} 28 \text{x} 28$} (input);
    \node   [bn] (bn1)   [right of=conv1]      {};
    %   edge [pre] (conv1);
    \node   [relu] (relu1) [right of=bn1]      {};
    %   edge [pre] (bn1);
    \node   [do] (do1)   [right of=relu1]      {};
    %   edge [pre] (relu1)
    \node   [conv] (conv2) [right=1.5cm of do1]{}
        edge [pre] node {$b \text{x} 16 \text{x} 10 \text{x} 10$} (do1);
    \node   [bn] (bn2)   [right of=conv2]      {};
    %   edge [pre] (conv2);
    \node   [relu] (relu2) [right of=bn2]      {};
    %   edge [pre] (bn2);
    \node   [do] (do2)   [right of=relu2]      {};
    %   edge [pre] (relu2);
    \node   [conv] (conv3) [right=1.5cm of do2]{}
        edge [pre] node {$b \text{x} 32 \text{x} 10 \text{x} 10$} (do2);
    \node   [bn] (bn3)   [right of=conv3]      {};
    %   edge [pre] (conv3);
    \node   [relu] (relu3) [right of=bn3]      {};
    %   edge [pre] (bn3);
    \node   [do] (do3)   [right of=relu3]      {};
    %   edge [pre] (relu3);
    \node   [avg] (avg1)  [right=1.5cm of do3] {}
        edge [pre] node {$b \text{x} 64 \text{x} 10 \text{x} 10$} (do3);
    \node   [fc]  (fc1)  [right=1.5cm of avg1] {}
        edge [pre] node {$b \text{x} 64 \text{x} 5 \text{x} 5$} (avg1);
    \node   [relu]  (relufc)  [right of=fc1] {};
    \node   [fc] (fc2) [above right=-0.5cm and 1.5cm of relufc] {}
        edge [pre] node [sloped,above] {$b \text{x} 1 \text{x} 40$} (relufc);
    \node   [fc] (fc3) [below right=-0.5cm and 1.5cm of relufc] {}
        edge [pre] node [sloped,below] {$b \text{x} 1 \text{x} 40$} (relufc);
    \node   [latent code]   (lc) [right=3cm of fc1] {};
    \draw ($ (fc2)!0.5!(fc3) + (0.1cm,0cm) $) -- 
        node [above,text width=1cm,text centered] {reparam\\trick}
        node [below] {$b \text{x} 1 \text{x} 20$}
        ($ (lc) + (-0.25cm,0) $) [arrow,>=stealth',semithick];
    \draw ($ (lc) + (0.25cm,0cm) $) -- ($ (lc) + (0.75cm,0cm) $)
        -- ($ (lc) + (0.75cm,-1.5cm) $)
        -- ($ (conv1) + (-0.6cm, -1.5cm) $)
        -- ($ (conv1) + (-0.6cm, -3cm) $)
        -- ($ (conv1) + (-0.1cm, -3cm) $) [arrow,>=stealth',semithick];
    \node [fc] (fc4) [below=2cm of conv1] {};
    \node [fc] (fc5) [right of=fc4] {};
    \node [convt] (ct1) [right=1.5cm of fc5] {}
        edge [pre] node {$b \text{x} 64 \text{x} 5 \text{x} 5$} (fc5);
    \node   [bn] (bn4)   [right of=ct1]      {};
    %   edge [pre] (ct1);
    \node   [relu] (relu4) [right of=bn4]      {};
    %   edge [pre] (bn4);
    \node   [do] (do4)   [right of=relu4]    {};
    %   edge [pre] (relu4);
    \node   [convt] (ct2)   [right=1.5cm of do4]      {}
        edge [pre] node {$b \text{x} 64 \text{x} 10 \text{x} 10$} (do4);
    \node   [bn] (bn5)   [right of=ct2]      {};
    %   edge [pre] (ct2);
    \node   [relu] (relu5) [right of=bn5]      {};
    %   edge [pre] (bn5);
    \node   [do] (do5)   [right of=relu5]    {};
    %   edge [pre] (relu5);
    \node   [convt] (ct3)   [right=1.5cm of do5]      {}
        edge [pre] node {$b \text{x} 32 \text{x} 10 \text{x} 10$} (do5);
    \node   [bn] (bn6)   [right of=ct3]      {};
    %   edge [pre] (ct3);
    \node   [relu] (relu6) [right of=bn6]      {};
    %   edge [pre] (bn6);
    \node   [do] (do6)   [right of=relu6]    {};
    %   edge [pre] (relu6);
    \node   [convt] (ct4)   [right=1.5cm of do6]      {}
        edge [pre] node {$b \text{x} 16 \text{x} 10 \text{x} 10$} (do6);
    \node   [output] (out)  [right=1.5cm of ct4]      {} 
        edge [pre] node {$b \text{x} 1 \text{x} 28 \text{x} 28$} (ct4);
\end{tikzpicture}
\end{center}
\caption{Arquitectura VAE para dataset MNIST}%
\label{fig:mnist_vae}
\end{figure}
\shorthandon{<}
\shorthandon{>}



\section{Discusión}%
\label{sec:discusion}

\section{Conclusión}%
\label{sec:conclusion}

\addcontentsline{toc}{section}{\refname}

\printbibliography

\end{document}
