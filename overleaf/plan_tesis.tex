\documentclass[spanish]{article}
\usepackage[utf8]{inputenc}
\usepackage[T1]{fontenc}
\usepackage{graphicx}
\usepackage[spanish]{babel}
\usepackage[a4paper, margin=1 in]{geometry}
\usepackage{indentfirst}
\graphicspath{ {./images/} }

\begin{document}

\begin{titlepage}
    \begin{center}

    \huge
    \textbf{Plan de Tesis de Maestría en Explotación de Datos y Descubrimiento del Conocimiento}
    
    \vspace{0.5cm}
    
    \vfill
    \Large
    Tesis a ser presentada para obtener el título de\\
    Magíster en Explotación de Datos y Descubrimiento del Conocimiento
    
    \vspace{0.5cm}
    
    \includegraphics[width=0.4\textwidth]{images/logo_uba.png}
    
    \vspace{0.5cm}
    \large
    Facultad de Ciencias Exactas y Naturales\\
    Maestría en Explotación de Datos y Descubrimiento del Conocimiento\\
    
    \vspace{0.5cm}
    
    \includegraphics[width=0.4\textwidth]{images/logo_maestria_dm.png}

    \vspace{0.5cm}
    
    Autor: Pablo Santoro
    
    \vspace{0.5cm}
    
    Director de Tesis:\\
    Dr. Emmanuel Iarussi
    
    {\selectlanguage{spanish}\today}
    
    \end{center}
\end{titlepage}


\section{Resumen}
El presente plan de trabajo propone como proyecto de tesis la implementación de una arquitectura de red neuronal
profunda a modelos tridimensionales. El objetivo principal es desarrollar un modelo generativo \cite{Foster2019}
que permita explorar el espacio de forma de las figuras de entrenamiento y que, además, sea capaz de generar
nuevas variantes para estos objetos. En el contexto de este problema, es deseable que las características
morfológicas y estéticas de los modelos generados puedan ser controladas mediante parámetros independientes. Nos
proponemos evaluar distintas arquitecturas tradicionales de la literatura en redes neuronales profundas y bases
de datos tanto propias como existentes \cite{G2L18} para el análisis de los modelos generativos que deriven de
esta tesis. Se probará, asimismo, la capacidad de aislar los parámetros constitutivos de las figuras por parte
de estas redes para poder comprender mejor las distintas características capturadas por las redes y para poder
modificarlos de forma independiente al resto de los otros parámetros, asociados estos a características
específicas de las figuras.

\section{Tema de investigación}
El éxito de los modelos de redes neuronales profundas aplicados a imágenes ha llevado a muchos investigadores a
intentar trasladar estas técnicas a datos tridimensionales. Además de los habituales casos de clasificación
\cite{qi2017pointnet, savva2016shrec16} y reconstrucción \cite{fan2017point, arsalan2017synthesizing}, hay cada
vez más trabajos \cite{riegler2017octnet, girdhar2016learning} que emplean diversas arquitecturas de red neuronal
para entrenar modelos generativos: se pretende, no sólo la correcta reconstrucción y/o clasificación de los datos
sino también poder tomar nuevas muestras de la distribución generadora de los datos. Esto es, el modelo
generativo tiene un componente estocástico que aproxima a la implícita función generadora de datos de
entrenamiento, con la cual luego se pueden muestrear nuevos objetos. De esta manera se pueden generar datos no
observados por el modelo, sino inferidos a partir de una serie reducida de parámetros denominados
\textit{latentes} que describen el espacio en el que se encuentran los datos y que se suelen organizar en una
representación vectorial multidimensional.

Algunas de las arquitecturas actualmente más utilizadas en el campo de redes generativas profundas son los
\textit{Variational Autoencoders} (VAE) \cite{Kingma2013}, y \textit{Redes Generativas Antagónicas}
(Generative Adversarial Networks - GAN) \cite{Goodfellow2014} (Figura \ref{GAN}). Estas serán las arquitecturas
que se usarán en el desarrollo de la tesis, observando su performance tanto en la reconstrucción de datos como en
la obtención de nuevos y en la capacidad de generar parámetros independientes con los cuales manipular distintas
características de las figuras generadas por ellos asociadas en los códigos latentes generados.

\begin{figure}[h]
\includegraphics[height=4cm]{images/gan_generator.png}
\centering
\caption{Ejemplo de generador de red GAN en \cite{Wu2016} para la generación de objetos}
\label{GAN}
\end{figure}

Muchas de las arquitecturas de red propuestas hasta el momento
\cite{Gao2019, Park2019, Olszewski2019, Li2019, Muralikrishnan2019, Yin2019} intentan lograr una reconstrucción
fidedigna de los datos de entrada a las redes a la vez que inferir una serie de parámetros que, a posteriori, se
puedan muestrear para generar nuevos datos. Asimismo, otro de los objetivos que se suelen perseguir en este tipo
de modelos es lograr que los parámetros latentes sean independientes, lo que significa que las alteraciones de
algunas componentes del vector latente sólo afecten ciertas partes de las figuras y no a otras, con el objetivo
de dejar constantes algunos aspectos y modificar otros. Este problema es particularmente desafiante, ya que
implica que la red debe aprender automáticamente la semántica de los modelos tridimensionales que intenta
reconstruir, diferenciando su forma y estilo, y asignando diferentes parámetros para controlar sus variaciones.
Los intentos por resolver este problema abarcan un amplio abanico de estrategias, desde entrenar redes
individuales en datos previamente segmentados \cite{Li2019} a modificar ciertos aspectos de la función de pérdida
y/o de la arquitectura de la red para lograr que la red identifique los parámetros latentes
\cite{Yin2019, Higgins2017}. En esta tesis proponemos utilizar diversas estrategias operando sobre la
arquitectura de la red neuronal profunda con el doble objetivo de generar parámetros independientes para el
muestreo de nuevos objetos sin entrenar más de una red para lograrlo. Ésta es la diferencia principal con la
literatura actual, en donde no alcanzan las arquitecturas existentes a generar un código latente con parámetros
independientes entrenando sólo un modelo.

En cuanto a datos tridimensionales existen distintos tipos de datos que se pueden considerar. Los más frecuentes
son las mallas poligonales \cite{Groueix2018}, nubes de puntos en espacio tridimensional \cite{qi2017pointnet},
voxeles \cite{girdhar2016learning} o incluso vistas ya proyectadas en 2D del objeto  \cite{Muralikrishnan2019}
(Figura \ref{ShapeUnicode}). Cada una de estas representaciones presenta ventajas y desventajas. En esta tesis
nos proponemos trabajar con representaciones voxelizadas, principalmente con el fin de aprovechar la gran
cantidad de modelos de disponibilidad pública disponibles en este formato. Además, en nuestro contexto, la 
representación volumétrica mediante voxels tiene otras ventajas tales como la fácil generación de datos
sintéticos para testeo y por último, la fácil y natural adaptación de las redes profundas convolucionales
(Convolutional Neural Networks, o CNN por sus siglas en inglés) existentes para imágenes en dos dimensiones a
datos tridimensionales. No obstante, esta elección puede presentar problemas de escalabilidad, ya que incluso
a resoluciones bajas la cantidad de datos crece de forma cúbica. Por ende, trabajaremos en bajas resoluciones
de manera que sea posible entrenar nuestras arquitecturas de red en tiempos razonables. Además, las
representaciones voxelizadas de objetos representan de manera correcta la forma y las distintas partes de las
que se componen.

\begin{figure}[h]
\includegraphics[width=8cm]{images/shape_unicode.png}
\centering
\caption{Ejemplos de VAE para generar una traducción entre distintas formas de representación de datos en \cite{Muralikrishnan2019}}
\label{ShapeUnicode}
\end{figure}

% REVISAR ESTO BIEN

La importancia de este tipo de redes generativas se debe a la vasta posibilidad de aplicaciones que pueden
tener este tipo de arquitecturas, desde interpolaciones en un espacio continuo de parámetros que permitan la
generación de figuras que no se han observado en tiempo de entrenamiento como también poder aislar diversos
aspectos de los datos para luego poder manipular las características de diseño de las figuras sin conocer
previamente una función generadora de objetos, sino simplemente teniendo disponibles datos sobre los cuales
entrenar arquitecturas de este tipo. Esto puede dar lugar a aplicaciones en el diseño industrial de productos
manufacturados como también aportar herramientas inteligentes a las interfaces de generación y manipulación
de contenido digital 3D como por ejemplo Blender \footnote{https://www.blender.org} o SolidWorks
\footnote{https://www.solidworks.com}.

\section{Estado del arte}
Los autoencoders son redes neuronales compuestas por dos partes. La primera consiste en el encoder, que
comprimen los datos en una representación vectorial de dimensionalidad reducida, y el decoder, que toma dicha
representación vectorial y reconstruye los datos de entrada a la red \cite{Foster2019}. Estas redes se
entrenan minimizando una función de pérdida entre los datos de entrada y de salida, por ende, el
entrenamiento de autoencoders es no supervisado, teniendo como objetivo la correcta reconstrucción de los
datos de entrada. Una de las características de los autoencoders es que permiten la proyección de datos en un
subespacio de menor dimensionalidad, los cuales representan una generalización, al capturar variables
latentes en el espacio de los datos.
Las arquitecturas VAE \cite{Kingma2013} y GAN \cite{Goodfellow2014} son la base para muchas variantes de
autoencoders aplicadas a modelos generativos de redes profundas para datos tridimensionales, incluso muchas
veces usadas de forma conjunta en redes que se dan en llamar VAE-GAN \cite{Li2019}. Es este último tipo de
arquitecturas y las VAE las que serán de mayor enfoque en el trabajo. VAE consiste en una modificación de
los autoencoders tradicionales: se agrega una nueva componente a la función de pérdida (que habitualmente
consta de componentes de errores de reconstrucción), la divergencia Kullback-Leibler. La idea es que cada
uno de los componentes del código latente se asemeje a una normal estándar, con lo que, cuanto más alta la
divergencia, más incide esta componente en la función de pérdida, y más se forzará la red a distribuir al
código latente como una normal multivariada estándar. Las arquitecturas GAN, por su parte, consisten en
entrenar dos redes profundas con objetivos antagónicos. Por un lado, la red Generadora tiene como misión
aprender a generar  contenido a partir de un código latente que se muestrea de forma aleatoria de una
distribución a priori, habitualmente normal estándar. El contenido generado (imágenes, modelos
tridimensionales, etc.) se somete luego al escrutinio de una segunda red, el Discriminador. Esta red, que
consiste en un clasificador, tiene como misión distinguir entre el contenido real (provenientes del
dataset) y falso (proveniente del Generador). Esto hace que estas redes se entrenen con objetivos
contrapuestos: la función de pérdida del generador consiste en incrementar el error del Discriminador,
y el Discriminador debe evitar que el Generador lo engañe \cite{goodfellow2016deep}. Luego del
entrenamiento, el Generador puede generar nuevos objetos a partir del muestreo de la distribución
utilizada. Generalmente, en los VAE-GAN, se usa una arquitectura VAE como generador y se adosa un
discriminador para detectar si una imagen es verdadera o falsa. Esto consigue, por lo general, una mayor
definición en características de alta varianza en el dataset (Figura \ref{PartVAE-GAN}).

\begin{figure}[h]
\includegraphics[height=5cm]{images/PartVAE-GAN.png}
\centering
\caption{Arquitectura VAE-GAN por parte en un dataset segmentado (izquierda) para reconstruir figuras ensamblando las partes (derecha) \cite{Li2019}}
\label{PartVAE-GAN}
\end{figure}

Si bien se puede lograr una mejor reconstrucción usando autoencoders tradicionales, la ventaja de usar
VAE es el hecho de obtener un espacio continuo en el código latente y variables del mismo con una
distribución cercana a una normal estándar, lo que facilita luego el poder muestrear fácilmente nuevo
contenido de ese espacio. No obstante, se ha observado \cite{Higgins2017, burgess2018understanding}
que el código latente formado por VAE tiene sus componentes muy correlacionados, lo que provoca que
variar una sola de estas componentes impacta en múltiples cambios semánticamente relacionados en las
figuras generadas. Esta característica dificulta el uso de este código para controlar las mencionadas
características de forma independiente. En la literatura, se propone resolver este problema modificando
la función de pérdida \cite{Higgins2017} o modificando la arquitectura \cite{Li2019, Yin2019}. De esta
forma se consiguen parámetros más independientes a costa de perder un poco de calidad en la
reconstrucción de objetos. En \cite{Yin2019} se genera un tipo de código llamado ``overcomplete'', el
cual está compuesto por salidas intermedias de las capas de la red de entrada, lo cual genera un
``disentangling'' implícito en estas partes a la hora de la reconstrucción de los datos de entrada.
\cite{Li2019}, por otro lado, adopta la estrategia de entrenar una red VAE-GAN por cada una de las
partes de los objetos de un dataset previamente segmentado, en donde a la salida de estas redes las
partes reconstruidas entran en una red que es la encargada del ensamblaje de las partes. La desventaja
fundamental de esta forma de trabajo es el entrenar varias redes para reconstruir una misma forma, por
lo que la complejidad del modelo escala rápido. \cite{Park2019} usa una simplificación de autoencoders
llamada autodecoders, que no entrenan un encoder y el código latente se adapta mediante el uso del
algoritmo de propagación hacia atrás. En este caso, la arquitectura propuesta aproxima una función
signo, la cual reconstruye figuras a partir de datos en forma de puntos muestreados cercanos a la
superficie de un objeto. Usando transformaciones en el código latente y en la orientación de los
objetos de entrenamiento, \cite{Olszewski2019} propone una arquitectura que puede anclar en el código
latente las rotaciones de un objeto, pudiendo generar nuevas vistas no usadas en los datos de
entrenamiento, logrando este objetivo usando vistas en imágenes bidimensionales de objetos
tridimensionales. \cite{Higgins2017} usa un coeficiente para modificar una componente de divergencia
Kullback-Leibler, el cual funciona como un hiperparámetro del modelo a ajustar para conseguir un mejor
``disentangling''.

\section{Objetivos}
\textbf{\textit{Objetivo general:}} Entender y proponer un modelo generativo de figuras tridimensionales
que generen un espacio de dimensiones latentes independientes entre sí con los cuales generar nuevas
muestras de figuras.

\textbf{\textit{Objetivos específicos:}}
\begin{enumerate}
   \item Implementar distintos modelos generativos para la generación de espacios de dimensiones latentes
	   con parámetros independientes.
   \item Entrenar las arquitecturas implementadas sobre datos sintéticos generados a partir de una
	   función generativa de datos tridimensionales conocida, con el fin de entender la capacidad de
	   aquellas de aproximar la función generadora utilizada.
   \item Evaluar el desempeño sobre los datos sintéticos y sobre las bases de datos existentes. 
\end{enumerate}

\section{Plan de Trabajo}
\textit{Etapas a cumplir según el comentado proyecto de investigación:}
\begin{enumerate}
   \item \textbf{Revisión bibliográfica.} Investigación del estado del arte sobre distintos tipos de redes
	   profundas utilizadas sobre datos bi y tridimensionales y los resultados conseguidos.
   \item \textbf{Recopilación / generación de datos.}Utilización de datos sintéticos y conjuntos de datos
	   utilizados en otros trabajos del área.
   \item \textbf{Evaluación de distintas arquitecturas sobre datos sintéticos.}
   \item \textbf{Evaluación de distintas arquitecturas sobre datos utilizados en otros trabajos del área.}
	   Especialmente evaluar las arquitecturas utilizadas en el punto anterior en estos datos.
   \item \textbf{Análisis de los resultados obtenidos y preparación del informe final.}
\end{enumerate}
\begin{center}
   \includegraphics[width=1\textwidth]{images/gantt.png}
\end{center}

\bibliographystyle{plain}
\addcontentsline{toc}{section}{\refname}
\bibliography{tesis_dm}

\end{document}

